% !TEX root = main.tex
\chapter{Einleitung}
Hier sollen die Grundlagen dafür gelegt werden, auch Operatoren mit variablen Koeffizienten richtig behandeln zu können. Die Darstellung basiert auf \cite{Hormander:1965}, \cite{Hormander:1966}, sowie \cite[Kapitel 18]{Hormander:1985}. Zuerst verallgemeinern wir den Begriff des Differentialoperators soweit, daß wir auch Inverse und allgemeinere Funktionen von solchen Operatoren behandeln können. 

\section{Operatoren und Symbole}
Sei $\Omega$ eine Mannigfaltigkeit. Differentialoperatoren auf $\Omega$ können dann in lokalen Karten definiert werden oder global durch ihre Eigenschaften charakterisiert werden. Die bekannteste davon ist, daß eine stetige lineare Abbildung $P:\rmC_0^\infty(\Omega)\to\rmC^\infty(\Omega)$ genau dann Differentialoperator ist, wenn $P$ lokal ist, also wenn
\begin{equation}
  \forall f\in\rmC_0^\infty(\Omega)\quad:\quad \supp Pf \subseteq \supp f
\end{equation}
gilt. Für Verallgemeinerungen brauchbarer ist folgende Charakterisierung:

\begin{lem}
Eine  lineare Abbildung   $P:\rmC_0^\infty(\Omega)\to\rmC^\infty(\Omega)$ ist genau dann ein Differentialoperator der Ordnung $m$, wenn für alle $f\in\rmC_0^\infty(\Omega)$ und alle $g\in\rmC^\infty(\Omega)$ die Funktion 
\begin{equation}\label{eq:6:6.2}
\e^{-\i\lambda g} P(f\e^{\i\lambda g}) = \sum_{j=0}^m P_j(f,g) \lambda^j
\end{equation}
ein Polynom vom Grad $m$ in $\lambda$ ist.
\end{lem}
\begin{proof}
Die Hinrichtung ist klar. Zum Beweis der Rückrichtung sei $x'\in\Omega$ ein Punkt und $f\in\rmC_0^\infty(\Omega)$ mit $f=1$  in einer Umgebung $\omega$ von $x'$. Sei weiter $\xi\in\R^n$ und $g(x) = x\cdot\xi$. Dann folgt aus \eqref{eq:6:6.2}
\begin{equation}
    \e^{-\i \lambda x\cdot\xi} P ( f\e^{\i \lambda x\cdot\xi}) = \sum_{j=0}^m p_j(f;x,\xi) \lambda^j
\end{equation}
mit $\rmC^\infty$-Funktionen auf $\omega\times\R^n$ als Koeffizienten $p_j(f;x,\xi)$. Da auch
$p_j(f;x,\lambda\xi)=p_j(f;x,\xi)\lambda^j$ gilt, ist $p_j(f;x,\xi)$ homogen vom Grad $j$. Die einzigen auf ganz $\R^n$ glatten homogenen Funktionen sind Polynome,
$p_j(f;x,\xi) = \sum_{|\alpha|=j} a_\alpha(f;x) \xi^\alpha$. Sei nun $u\in\rmC_0^\infty(\omega)$. Dann gilt mit der Fourierschen Inversionsformel
\begin{equation}
    u(x) = (2\pi)^{-n/2} \int \e^{\i x\cdot\xi} \widehat u(\xi)\d\xi
\end{equation}
und da $u=fu$ folgt insbesondere 
\begin{align}
    Pu(x) &= P(fu)(x) = (2\pi)^{-n/2} \int P(f \e^{\i x\cdot\xi}) \widehat u(\xi)\d\xi \notag \\&= \sum_{j=0}^m (2\pi)^{-n/2} \int \e^{\i x\cdot\xi} p_j(f;x,\xi) \widehat u(\xi)\d\xi
    = \sum_{|\alpha|\le m} a_\alpha(f;x)\D^\alpha u.
\end{align}
Mit Linearität folgt die Behauptung.
\end{proof}
