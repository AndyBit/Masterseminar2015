% !TEX root = main.tex
\chapter{Minimale Operatoren}

\def\N{\mathbb N}
\def\R{\mathbb R}
\def\C{\mathbb C}
\def\mcP{\mathcal P}
\def\mcQ{\mathcal Q}
\def\mcR{\mathcal R}
\def\pr{\prime}
\def\tr{\mathrm{T}}
\def\al{\alpha}
\def\be{\beta}
\def\de{\delta}
\def\ep{\epsilon}
\def\ga{\gamma}
\def\x{\xi}
\def\y{\eta}
\def\z{\zeta}
\def\Om{\Omega}
\def\infi{\infty}
\def\d{\partial}
\def\df{\mathrm{d}}
\def\til#1{\widetilde{#1}}
\def\mto{\mapsto}
\def\ti{\times}
\def\abs#1{\lvert#1\rvert}
\def\norm#1{\lVert#1\rVert}
\def\Lap{\Delta}
\def\cc#1{\overline{#1}}
\def\ul#1{\underline{#1}}
\def\ska#1#2{\langle#1,#2\rangle}
\def\Ska#1#2{\left\langle#1,#2\right\rangle}
\def\div{\operatorname{div}}
\def\diam{\operatorname{diam}}

Als erste Etappe in diesem Abschnitt führen wir den Vergleich der Stärke (Definition \ref{df:1:1.2})
zweier Differentialperatoren mit konstanten Koeffizienten,
auf einen Vergleich der Symbole zurück.

\section{Die Stärke von Ableitungspolynomen}

Es seien in diesem Abschnitt stets $\mcP=P(D)$, $\mcQ=Q(D)$ und $\Om$ ein beschränktes Gebiet.
Ziel ist die Charaktierisierung der Aussage ``$\mathcal Q$ ist schächer als $\mcP$''
durch eine Ungleichung der Form
\begin{equation}\label{eq:2:1}
\sup_{\x\in\R^n}\frac{\til{Q}(\x)}{\til{P}(\x)}<\infi,
\end{equation}
wobei $\til{P}$ und $\til{Q}$ geeignete Regularisierungen von $P$ und $Q$ sind.
Beachte, dass die reellen Nullstellen von $P$ Probleme machen würden.

Im eindimensionalen Fall ist bekannt, dass $Q(D)$ genau dann schwächer wie $P(D)$ ist,
wenn $P(\z)$ höheren oder gleichen Grad wie $Q(\z)$ hat.
%Zum Beweis genügt die mehrfache Anwendung des Theorems \ref{thm:1:1.8}
%auf den Operator $D$ und der Dreiecksungleichung.
Als Regularisierungen könnte man $\til{P}(\z)=\sqrt{1+\abs{P(\z)}^2}$
und $\til{Q}(\z)=\sqrt{1+\abs{Q(\z)}^2}$ verwenden.
Daß dies schon im zweidimensionalen nicht mehr Funktioniert
hängt damit zusammen, dass $\abs{P^{(\al)}(\z)}/\sqrt{1+\abs{P(\z)}^2}$
nun im allgemeinen nicht mehr beschränkt ist.
Man betrachte zum Beispiel $P(\z)=\z_1\z_2$ und $P^{(1)}(\z)=\z_2$.

Es zeigt sich, dass die Regularisierungen
\begin{equation}\label{eq:2:2}
\til{P}(\z):=\sqrt{\sum_\al\abs{P^{(\al)}(\z)}^2},
\quad\text{f.a.}~\z\in\C^n,
\end{equation}
eine natürliche Wahl sind.
Also der Absolutbetrag des Vektors $(P^{(\al)}(\z))_\al\in\C^M$.
Beachte, dass für $\x\in\R^n$ der Ausdruck
$\til{P}(\x)^2$ ein reelles Polynom ohne Nullstellen ist.

\begin{thm}\label{thm:2:2.1}
Der Differentialoperator $\mcQ$ ist genau dann schwächer wie $\mcP$,
wenn
\begin{equation}\label{eq:thm:2:2.1}
\sup_{\x\in\R^n}\frac{\til{Q}(\x)}{\til{P}(\x)}<\infi
\end{equation}
erfüllt ist.
\end{thm}

Wir zeigen zunächst, wie in beiden Richtungen abgeschätzt wird.
Die Hinrichtung ist straight forward,
für die Rückrichtung benötigen wir ein Lemma,
dass wir später beweisen.

\begin{proof}
{\em Hinrichtung}:
Sei $\mcQ$ schwächer wie $\mcP$.
Dann gilt für ein $C\in(0,\infi)$ die Ungleichung
\begin{equation}\label{eq:2:4}
\norm{\mcQ u}\leq C(\norm{\mcP u}^2+\norm{u}^2),\quad\text{f.a.}~u\in C^\infi_0(\Om).
\end{equation}
Für jedes $\x\in\R^n$ und ein festes $\psi\in C^\infi_0(\Om)$, $\psi\neq0$,
definieren wir $\psi_\x(x):=\psi(x)\e^{\ska{x}{\x}}$, f.a.~$x\in\R^n$.
Dies bringt durch die Formel von Leibniz die Ableitungen von $P(\x)$ ins Spiel
\begin{equation}\label{eq:2.5}
P(D)\psi_\x(x)=\e^{\ska{x}{\x}}\sum_\al P^{(\al)}(\x)\frac{D_\al\psi(x)}{\al!}.
\end{equation}
Schreiben wir kurz
\begin{equation}\label{eq:2.6}
\Psi_{\al\be}:=\frac1{\al!\be!}\spro{D_\al\psi}{D_\be\psi},
\end{equation}
so liest sich die Ungleichung \eqref{eq:2:4} mit $u=\psi_\x$ als
\begin{equation}\label{eq:2.7}
\sum_{\al,\be}Q^{(\al)}(\x)\bar{Q}^{(\be)}(\x)\Psi_{\al\be}
\leq C\left(\sum_{\al,\be}P^{(\al)}(\x)\bar{P}^{(\be)}(\x)\Psi_{\al\be}+\Psi_{00}\right).
\end{equation}
Da $\psi_\x\in C^\infi_0$, $\psi_\x\neq0$ gilt, sind $\norm{\mcP\psi_\x}$ und $\norm{\mcQ\psi_\x}$ stets positiv.
Somit ist $\left(\Psi_{\al\be}\right)_{\al,\be}$ auch die gram'sche Matrix
eines Skalarproduktes auf einem $\C^M$.
Es gibt also $\Psi_-,\Psi_+\in(0,\infi)$, so dass
\begin{equation}\label{eq:2.8}
\abs{\z}^2\Psi_-\leq\sum_{\al,\be}\z_\al\z_\be\Psi_{\al\be}\leq\abs{\z}^2\Psi_+,
\quad\text{f.a.}~\z\in\C^M.
\end{equation}
Die Ungleichungen \eqref{eq:2.7} und \eqref{eq:2.8} kombinieren sich zu
\begin{equation}\label{eq:2.9}
\til{Q}(\x)^2\Psi_-C\leq\Psi_+\til{P}(\x)^2,
\quad\text{f.a.}~\x\in\R^n,
\end{equation}
und dies ist gerade die Ungleichung \eqref{eq:thm:2:2.1}.

{\em Rückrichtung}:
Sei die Ungleichung \eqref{eq:thm:2:2.1} erfüllt, d.h.
\begin{equation}\label{eq:2.10}
\til{Q}(\x)\leq C\til{P}(\x),
\quad\text{f.a.}~\x\in\R^M,
\end{equation}
mit $C\in(0,\infi)$.
Wir schätzen wie folgt ab
\begin{equation}\label{eq:2.11}
\norm{Q(D)u}^2=\norm{Q\cdot\hat{u}}^2\leq C\norm{\til{P}\cdot\hat{u}}^2
\leq C\sum_{\al}\norm{P^{(\al)}\cdot\hat{u}}^2=C\sum_\al\norm{P^{(\al)}(D)u}^2.
\end{equation}
Um den Beweis abzuschließen fehlte also nur noch eine Ungleichung
\begin{equation}
\norm{P^{(\al)}(D)u}\leq C_{\al,\mcP}\norm{P(D)u},
\quad\text{f.a.}~u\in C^\infi_0(\Om),
\end{equation}
mit $C>0$.
Diese werden wir mit der Methode der Energieintegrale herleiten (Korollar \ref{cor:2:2.5}).
\end{proof}

Der Beweis zeigt,
dass in Formel \eqref{eq:thm:2:2.1} $\til{Q}$ durch $Q$ ersetzt werden kann.

\section{Energieintegrale}

Da wir eine Methode brauchen um Skalarprodukte
mit verschiedenen Differentialoperatoren abzuschätzen,
macht es Sinn, quadratische Differentialformen $F$, definiert durch
\begin{equation}
F(D,\bar{D})u\bar{u}:=\sum_{\al,\be}a_{\al\be}(D^\al u)(\cc{D^\be u}),
\quad\text{f.a.}~u\in C^\infi_0(\Om),
\end{equation}
mit $a_{\al\be}\in\C$, genauer zu betrachten.
Uns interessieren die Terme
\begin{equation}\label{eq:2:qf}
\int F(D,\bar{D})u(x)\bar{u}(x)\df x=\int F(\x,\x)\abs{\hat{u}(\x)}^2\df \x,
\quad\text{f.a.}~u\in C^\infi_0(\Om),
\end{equation}
die quadratische Formen \eqref{eq:2:qf} definieren.
Die Formen sind, wie die Gleichung zeigt,
durch das Symbol $F(\z,\cc{\z})$, $\z\in\C^n$ bestimmt.
Umgekehrt sind die reellen Werte des Symbols
durch die Form \eqref{eq:2:qf} bestimmt,
wie folgendes Lemma zeigt.
Im allgemeinen können verschiedene Symbole
die gleiche Form \eqref{eq:2:qf} definieren.

\begin{lem}\label{lem:2:qfsym}
Sei $\Om$ ein beliebiges Gebiet.
Die Gleichung
\begin{equation}\label{eq:lem:2:qfsym}
\int F(D,\bar{D})u\bar{u}\df x=0,
\quad\text{f.a.}~u\in C^\infi_0(\Om),
\end{equation}
gilt genau dann, wenn $F(\x,\x)=0$, f.a.~$\x\in\R^n$.
\end{lem}
\begin{proof}
Es gelte \eqref{eq:lem:2:qfsym}.
Sei $u_\y(x):=u(x)\e^{\ska{x}{\y}}$ für ein festes $u\in C^\infi_0(\Om)$, $u\neq0$.
Dann ist
\begin{equation}
\int F(D,\bar{D})u_\y\bar{u_\y}\df x=\int F(\x,\x)\abs{\hat{u}(\x-\y)}^2\df \y=\int F(\x+\y,\x+\y)\abs{\hat{u}(\x)}^2\df\x=:q(\y),
\end{equation}
für alle $\y\in\R^n$.
Auf der rechten Seite steht wieder ein Polynom $q(\y)$ in $\y\in\R^n$
und nach Voraussetzung müssen dessen Koeffizienten verschwinden.
Beachten wir, dass
\begin{equation}
(\x+\y)^\al=\sum_{\be+\ga=\al}\begin{pmatrix}\al\\\be,\ga\end{pmatrix}\x^\be\y^\ga=\y^\al+\dots
\end{equation}
so sehen wir, dass die führenden Terme von $q(\y)$
bis auf den Faktor $\norm{u}^2\neq0$ den führenden Termen von $F(\x,\x)$ entsprechen.
Wäre also $F(\x,\x)\neq0$, so hätten wir einen Widerspruch.

Die Rückrichtung sahen wir bereits, siehe Gleichung \eqref{eq:2:qf}.
\end{proof}

Unser Ziel ist es, Ableitungen von Ableitungspolynomen zu kontrollieren.
Dazu betrachten wir Ableitungen quadratischer Differentialformen.
Wir beginnen mit dem Ausdruck
\begin{equation}
\frac{\d}{\d x_k}\left[F(D,\bar{D})u\bar{u}\right]=\i\left[(D_k-\bar{D}_k)F(D,\bar{D})\right]u\bar{u}.
\end{equation}
Für einen Vektor $\ul{G}=(G_k)_{k=1,\dots,n}$, bestehend aus quadratischen Differentialformen $G_k$,
erhält man für die Divergenz die Formel
\begin{subequations}
\begin{equation}\label{eq:2.15a}
\div(\ul{G}(D,\bar{D})u\bar{u})=\sum_{k=1}^n\frac{\d}{\d x_k}\left[G_k(D,\bar{D})u\bar{u}\right]=F(D,\bar{D})u\bar{u},
\end{equation}
wobei
\begin{equation}\label{eq:2.15b}
F(\z,\bar{\z})=\i\sum_{k=1}^n(\z_k-\bar{\z}_k)G_k(\z,\bar{\z})=-2\sum_{k=1}^n\y_kG_k(\z,\bar{\z}),
\end{equation}
\end{subequations}
wenn wir $\z=\x+\i\y$ schreiben.
\begin{lem}\label{lem:2:2.2}
Ein Polynom $F(\z,\bar{\z})$ in $\z=\x+\i\y$ und $\bar{\z}=\x-\i\y$
kann genau dann in der Form \eqref{eq:2.15b} dargestellt werden,
wenn $F(\x,\x)=0$ für alle $\x=0$.

In diesem Fall gilt
\begin{equation}\label{eq:2.16}
G_k(\x,\x)=-\tfrac12\left.\frac{\d F(\x+\i\y,\x-\i\y)}{\d \y_k}\right\rvert_{\y=0}.
\end{equation}
\end{lem}
\begin{proof}
Die Hinrichtung ist offensichtlich.
Die Rückrichtung folgt aus der (reellen) Taylorentwicklung von $F(\x+\i\y,\x-\i\y)$ in $(\x,\y)\in\R^{2n}$.
Der Satz von Taylor angewendet auf variables $\y\in\R^n$ und festes $\x\in\R^n$ liefert schließlich \eqref{eq:2.16}.
\end{proof}

Die Polynome $G_k(\z,\bar{\z})$, $\z\in\C^n$, sind als ganzes sind nicht eindeutig festgelegt.
Doch aufgrund von Lemma \ref{lem:2:qfsym} stört das nicht.

Mit Hilfe einer speziellen quadratischen Differentialformen und obiger Formel
gewinnen wir folgende Ungleichung,
mit der wir die Ableitungen der Differentialoperatoren kontrollieren können.

\begin{lem}
Sei $B_k:=\sup_{x_k,y_k\in\Om}\abs{x_k-y_k}<\infi$.
Für Ableitungspolynome $P(D)$, $Q(D)$ gilt die Ungleichung
\begin{equation}\label{eq:lem:2:2.4}
\abs{\spro{P^{(k)}(D)u}{\bar{Q}(D)u}}\leq\norm{P(D)u}\left(\norm{\bar{Q}^{(k)}(D)u}+B_k\norm{\bar{Q}(D)u}\right),
\quad\text{f.a.}~u\in C^\infi_0(\Om).
\end{equation}
\end{lem}

\begin{proof}
Wir betrachten die quadratische Differentialform mit dem Symbol
\begin{equation}
F(\z,\bar{\z}):=P(\z)Q(\bar{\z})-Q(\z)P(\bar{\z}).
\end{equation}
Diese erfüllt offensichtlich $F(\x,\x)=0$, f.a.~$\x\in\R^n$.
Nach Lemma \ref{lem:2:2.2} und der Formel \eqref{eq:2.16}
erhalten wir, nach dem wir mit $-\i x_k$ multipliziert haben
\begin{equation}\label{eq:2:2.22}
-\i x_kF(D,\bar{D})u\bar{u}=-\i x_k\sum_{k=1}^n\frac{\d}{\d x_k}\left[G_k(D,\bar{D})u\bar{u}\right],
\end{equation}
mit
\begin{equation}
G_k(\x,\x)=-\i\left(P^{(k)}(\x)Q(\x)-Q^{(k)}(\x)P(\x)\right),
\quad\text{f.a.}~\x\in\R^n.
\end{equation}

Integrieren wir die Gleichung \eqref{eq:2:2.22},
so liefert eine partielle Integration auf rechten Seite den Term
\begin{align}
\i\int G_k(D,\bar{D})u\bar{u}\df x&=\int\left(P^{(k)}(D)Q(\bar{D})-P(D)Q^{(k)}(D)\right)u\bar{u}\df x\\
&=\spro{P^{(k)}(D)u}{\bar{Q}(D)u}-\spro{P(D)u}{\bar{Q}^{(k)}(D)u}.
\end{align}
Damit erhalten wir insgesamt
\begin{equation}
\spro{P^{(k)}(D)u}{\bar{Q}(D)u}
=\spro{P(D)u}{\bar{Q}^{(k)}(D)u}-i\int x_k\left(P(D)u\cc{\bar{Q}(D)u}-Q(D)\cc{\bar{P}(D)u}\right)\df x.
\end{equation}
Ohne Beschränkung der Allgemeinheit können wir das Gebiet $\Om$ so legen,
dass $\abs{x_k}\leq B_k/2$ für alle $x_k\in\Om$.
Man beachte dass Ableitungspolynome mit Translationsoperatoren vertauschen.
Schließlich erhalten wir mit Cauchy-Schwarz
und der Dreiecksungleichung die Ungleichung \eqref{eq:lem:2:2.4}.
\end{proof}

\begin{cor}\label{cor:2:2.5}
Ist $P(\x)$ vom Grad $m$ in $\x_k$, so gilt
\begin{equation}\label{eq:cor:2:2.5}
\norm{P^{(k)}(D)u}\leq mB_k\norm{P(D)u},
\quad\text{f.a.}~u\in C^\infi_0(\Om).
\end{equation}
\end{cor}
\begin{proof}
Setzen wir $\bar{Q}(\x)=P^{(k)}(\x)$, so erhalten wir aus \eqref{eq:lem:2:2.4} die Ungleichung
\begin{equation}\label{eq:2.28}
\norm{P^{(k)}(D)u}^2\leq\norm{P(D)u}\left(\norm{P^{(kk)}(D)u}+B_k\norm{P^{(k)}(D)u}\right),
\quad\text{f.a.}~u\in C^\infi_0(\Om).
\end{equation}
Wir gehen nun Induktiv in $m$ vor.
Für $m=1$ ist $P^{(kk)}=0$, $P^{(k)}\neq0$ und somit entspricht die Gleichung \eqref{eq:cor:2:2.5}
gerade der Gleichung \eqref{eq:lem:2:2.4} nach kürzen eines Faktors.
Angenommen das Korollar ist für $m-1$ erfüllt,
dann erhalten durch Kombination von \eqref{eq:2.28} und \eqref{eq:cor:2:2.5}
\begin{equation}
\norm{P^{(k)}(D)u}^2\leq\norm{P(D)u}\left((m-1)B_k\norm{P^{(k)}(D)u}+B_k\norm{P^{(k)}(D)u}\right),
\end{equation}
also ist auch der Induktionsschritt gezeigt.
\end{proof}

\begin{cor}\label{cor:2:2.6}
Für festes beschränktes Gebiet $\Om$ und beliebiges $\al\in\N^n_0$ gilt
\begin{equation}
\norm{P^{(\al)}(D)u}\leq C_\al\norm{P(D)u},
\quad\text{f.a.}~u\in C^\infi_0(\Om),
\end{equation}
mit $C_\al=m(m-1)\cdots(m-\abs{\al})B^\al$, wobei $B^\al=\prod_{l=1}^n B^{\al_l}_l$.
Die Konstante $C_\al$ hängt also nur vom Grad $m$ von $P(\x)$,
der Ableitungsordnung $\abs{\al}$ und der Ausdehnung des Gebiets $\Om$ ab.
\end{cor}

\begin{proof}
Mehrfache Anwendung von Korollar \ref{cor:2:2.5}.
\end{proof}

\section{Beispiele und spezielle Symbole}

Wir betrachten drei instruktive klassische Differentialoperatoren zweiter Ordnung.

\begin{exa}\label{exa:lap}
Wir betrachten das zum Laplace-Operator $\Lap$
korrespondierende Symbol $P(\x)=\x_1^2+\dots+\x_n^2$.
Das regularisierte Symbol ist gegeben durch
\begin{equation}
\abs{\til{P}(\x)}^2=(\x^2_1+\dots+\x^2_n)^2+4(\x_1^2+\dots+\x_n^2)+4n.
\end{equation}
Beachten wir, dass stets $(a+b)^2\leq2(a^2+b^2)$ für $a,b\in\R$,
so sehen wir, dass
\begin{equation}
\abs{\til{P}(\x)}^2\asymp(\x^2_1+\dots+\x^2_n)^2+1.
\end{equation}
Also ist der Laplaceoperator stärker wie alle Differentialoperatoren
mit Symbolen von gleichem oder kleinerem Grad.
Diese sind alle von der Form
\begin{equation}
Q(\x)=\x^\tr A\x+\ska{b}{\x}+c,
\end{equation}
mit einer Matrix $A\in\C^{n\ti n}$, $b\in\C^n$ und $c\in\C$,
wobei wir $A^\tr=A$ annehmen können.
Wir untersuchen, welche davon gleich stark wie der Laplaceoperator sind.
Es ist
\begin{align}
\abs{\til{Q}(\x)}^2&=\abs{\x^\tr A\x+\ska{b}{\x}+c}^2+\sum_{k=1}^n\abs{2e^\tr_kA\x+b_k}^2
+4\sum_{1\leq k\leq l\leq n}\abs{a_{kk}}^2,\\
&\asymp\abs{\x^\tr A\x}^2+\abs{\ska{b}{\x}}^2+\sum_{k=1}^n\abs{e^\tr_kA\x}^2+1.
\end{align}
Wegen
\begin{equation}
\frac{\abs{\ska{b}{\x}}^2+\sum_{k=1}^n\abs{e^\tr_kA\x}^2+1}{(\x^2_1+\dots+\x^2_n)^2+1}\to0,
\quad\text{falls}~\x\to0
\end{equation}
hängt alles vom Term $\abs{\x^\tr A\x}^2$ ab.
\end{exa}

\begin{exa}\label{exa:schroe}
Der Schrödingergleichung für ein freies Teilchen
\begin{equation}
\i\frac{\d}{\d t}\psi(x,t)=-\Lap\psi(x,t),\end{equation}
lässt sich das Symbol $P(\x)=\x^2_1+\dots+\x^2_{n-1}-\x_n$ zuordnen.

Wir bestimmen alle Operatoren gleicher stärke.
Das Symbol $Q(\x)$ ist genau dann schwächer wie $P(\x)$,
wenn
\begin{equation}\label{eq:2:subschroe}
\abs{Q(\x)}^2\apprle(\x^2_1+\dots+\x^2_{n-1}-\x_n)^2+\x^2_1+\dots+\x^2_{n-1}+1.
\end{equation}
Also hat $Q(\x)$ maximalen Grad 2 in $\x_1,\dots,\x_{n-1}$,
Grad 1 in $\x_n$ und somit von der Form
\begin{equation}
Q(\x)=a_0+\sum^n_{k=1}a_k\x_k+\sum^n_{k,l=1}a_{kl}\x_k\x_l,
\end{equation}
mit $a_{nn}=0$.
Wir betrachten die Gleichung \eqref{eq:2:subschroe}
für spezielle Vektoren:
\begin{equation}\label{eq:2:subschroespec}
\abs{Q(\x)}^2\apprle(\x^2_1+\dots+\x^2_{n-1}+1),
\quad\text{für alle}~\x\in\R^n~\text{mit}~\x_n=\x^2_1+\dots+\x^2_{n-1}.
\end{equation}
Da wir $\x_1,\dots,\x_{n-1}$ in \eqref{eq:2:subschroespec} frei wählen können folgt, dass
\begin{equation}
Q(\x)=a_0+\sum_{k=1}^{n-1}a_k\x_k+a_n(\x_n-\x^2_1-\dots-\x^2_{n-1}),
\end{equation}
mit beliebigen $a_0,a_1,\dots,a_n\in\C$.
Es ist $Q(\x)$ genau dann gleich Stark wie $P(\x)$,
wenn $a_n\neq0$.
\end{exa}

\begin{exa}\label{exa:heat}
Der Wärmeleitungsgleichung
\begin{equation}
\Lap T(x,t)=\frac{\d}{\d t}T(x,t),
\end{equation}
entspricht das Symbol $P(\x)=\x^2_1+\dots+\x^2_{n-1}+\i\x_n$.

Ein Symbol $Q(\x)$ ist genau dann stärker wie $P(\x)$, wenn
\begin{equation}
\abs{Q(\x)}^2\apprle(\x^2_1+\dots+\x^2_{n-1})^2+\x^2_1+\dots+\x^2_{n-1}+\x^2_n+1.
\end{equation}
Diese Ungleichung ist genau dann erfüllt, wenn
\begin{equation}
Q(\x)=a_0+\sum_{k=1}^na_k\x_k+\sum_{k,l=1}^{n-1}a_{kl}\x_k\x_l,
\end{equation}
mit beliebigen $a_{kl}\in\C$, $k,l=1,\dots,n-1$.
Das Symbol $Q(\x)$ ist genau dann gleich stark wie $P(\x)$,
wenn $a_n\neq0$ und die Matrix $(a_{kl})_{k,l=1,\dots,n-1}$ positiv ist
(diese wird ohne Beschränkung der Allgemeinheit symmetrisch gewählt).
\end{exa}

\begin{exa}\label{exa:hyper}
Als letztes Beispiel betrachten wir die Gleichung
\begin{equation}
\Lap_1u=\Lap_2u,
\end{equation}
mit $\Lap_1:=\d_1^2+\dots+\d_m^2$, $\Lap_2:=\d^2_{m+1}+\dots+\d^2_m$.
Das korrespondierende Symbol ist $P(\x)=\x^2_1+\dots+\x^2_m-\x^2_{m+1}-\dots-\x^2_n$.

Ein Symbol $Q(\x)$ ist genau dann gleich stark wie $P(\x)$,
wenn
\begin{equation}
\abs{Q(\x)}^2\apprle(\x^2_1+\dots+\x^2_m-\x^2_{m+1}+\dots+\x^2_n)^2+\x^2_1+\dots+\x^2_n+1.
\end{equation}
Setzen wir wie in Beispiel \ref{exa:schroe} $\x^2_1+\dots+\x^2_m=\x^2_{m+1}+\dots+\x^2_n$,
so erhalten wir für diese $\x$:
\begin{equation}
\abs{Q(\x)}^2\apprle\x^2_1+\dots+\x^2_n+1.
\end{equation}
Es folgt, dass $Q(\x)$ von der Form
\begin{equation}
Q(\x)=a_0+\sum_{k=1}^na_k\x^k+b(\x_1^2+\dots+\x^2_m-\x^2_{m+1}+\dots+\x^2_n),
\end{equation}
mit beliebigen $a_0,a_1,\dots,a_n,b\in\C$ sein muss
und $Q(\x)$ ist genau dann gleich stark wie $P(\x)$,
wenn $b\neq0$.
\end{exa}

\section{Weitere Vergleichsresultate}

Sehr analog zu Satz \ref{thm:2:2.1} lassen sich die regularisierten
Symbole $\til{P}$, $\til{Q}$ heranziehen
um die Kompaktheit der Abbildung
\begin{equation}\label{eq:2:map}
\mcR_{P_0}\to\mcR_{Q_0},\quad P(D)u\mto Q(D)u,
\end{equation}
zu charakterisieren (falls sie existiert).
Existiert die Abbildung \eqref{eq:2:map} und ist kompakt,
so nennen wir $Q(D)$ relativ kompakt zu $P(D)$.
Oben hatten wir charakterisiert,
wann die Abbildung \eqref{eq:2:map} existiert und stetig ist.

\begin{thm}
Der Operator $Q(D)$ ist genau dann relativ kompakt zu $P(D)$,
wenn
\begin{equation}\label{eq:2:tozero}
\frac{\til{Q}(\x)}{\til{P}(\x)}\to0,\quad\text{für}\quad\x\to\infi.
\end{equation}
\end{thm}
\begin{proof}
Sei die Gleichung \eqref{eq:2:tozero} erfüllt.
Wir wählen eine beliebige Folge $u_n\in C^\infi_0(\Om)$,
so dass 
\begin{equation}\label{eq:2:PDb}
\norm{P(D)u_n}\leq1.
\end{equation}
Wir zeigen, dass es eine Teilfolge $u_{n^\pr}$ gibt,
so dass $Q(D)u_{n^\pr}$ konvergiert.
Zunächst stellen wir fest, dass der Ausdruck $\til{Q}(\x)/\til{P}(\x)$
stets lokal-beschränkt ist.
Somit impliziert \eqref{eq:2:tozero} bereits \eqref{eq:thm:2:2.1}
und nach Satz \ref{thm:2:2.1} ist damit
\begin{equation}
\norm{Q(D)u_n}\leq C,
\quad\text{f.a.}~n\in\N,
\end{equation}
für ein $C\in(0,\infi)$.
Nach Hölder und der Stetigkeit der Fouriertransformation
von $L^1$ nach $L^\infi$ impliziert dies
\begin{equation}
\sqrt{2\pi}\norm{D^\al(Q(\x)u_n)}_\infi\leq\norm{x^\al Q(D)u_n}_1\leq\norm{Q(D)u_n}\norm{x^\al1_\Om}
\leq\norm{Q(D)u_n}\sqrt{\abs{\Om}}\tfrac{B^\al}{2^{\abs{\al}}}.
\end{equation}
In Kombination mit dem Riemann-Lebesgue-Lemma folgt,
dass alle Ableitungen von $Q(\x)\hat{u}_n(\x)$
gleichmäßig beschränkt sind und im unendlichen abfallend.
Insbesondere lässt sich eine lokal-gleichmäßig konvergente Teilfolge $Q(\x)\hat{u}_{n^\pr}(\x)$ auswählen.

Wir zeigen, dass dies unser Kandidat ist.
Nach Voraussetzung lässt sich zu jedem $\ep>0$ eine kompakte Menge $K$ wählen,
so dass $\abs{Q(\x)}/\til{P}(\x)<\ep$ für $\x$ im Komplement $K^{\mathrm c}$ von $K$.
Nun ist
\begin{align}
\int_{K^{\mathrm c}}\abs{Q(\x)}^2\abs{\hat{u}_{n^\pr}(\x)-\hat{u}_{m^\pr}(\x)}^2\df\x,
&\leq\ep^2\int\til{P}(\x)^2\abs{\hat{u}_{n^\pr}(\x)-\hat{u}_{m^\pr}(\x)}^2\df\x\\
&=\ep^2\sum_{\al\in\N^n_0}\norm{P^{(\al)}(D)(u_{n^\pr}-u_{m^\pr})}^2,\label{eq:2:Kcsum}
\end{align}
und die Summe in \eqref{eq:2:Kcsum} ist nach Korollar \ref{cor:2:2.6}
und \eqref{eq:2:PDb} beschränkt.
Weiter gilt
\begin{equation}
\int_K\abs{Q(\x)}^2\abs{\hat{u}_{n^\pr}(\x)-\hat{u}_{m^\pr}(\x)}^2\df\x\to0,
\quad\text{falls}~n^\pr~\text{und}~m^\pr\to\infi,
\end{equation}
aufgrund der lokal-gleichmäßigen Konvergenz.
Da $\ep>0$ beliebig gewählt werden konnte,
impliziert dies die Konvergenz von $Q(D)u_{n^\pr}$ in $L^2$.

Nehmen wir nun an $Q(D)$ ist kompakt relativ zu $P(D)$.
Dazu weisen wir nach, dass aus $\x_n\to\infi$ die Konvergenz $\til{Q}(\x_n)/\til{P}(\x_n)\to0$ folgt.
Wähle dazu $u\in C^\infi_0(\Om)$, $u\neq0$ fest und setze
\begin{equation}
u_n(x):=u(x)\frac{\e^{\i\ska{x}{\x_n}}}{\til{P}(\x_n)},\quad\text{f.a.}~u\in\R^n,
\end{equation}
für jedes $n\in\N$.
Wendet man die Leibnizsche Formel auf $u_n(x)$ an,
so erhält man
\begin{equation}
P(D)u_n(x)=\e^{\i\ska{x}{\x_n}}\sum_\al\frac{P^{(\al)}(\x_n)}{\til{P}(\x_n)}\frac{D^\al u(x)}{\al!}.
\end{equation}
Beachtet man, dass die Ausdrücke $P^{(\al)}/\til{P}$ stets beschränkt sind
und wendet wieder einmal Korollar \ref{cor:2:2.6} an,
so erhält man, dass
\begin{equation}\label{eq:2:PDunb}
\norm{P(D)u_n}\leq C,
\end{equation}
für ein $C\in(0,\infi)$.
Wir zerlegen nun
\begin{equation}\label{eq:2:kz}
\norm{Q(D)u_n-Q(D)u_m}^2=\norm{Q(D)u_n}^2+\norm{Q(D)u_m}^2-\de_{nm},
\end{equation}
wobei
\begin{equation}
\de_{nm}=2\Re\left\{\sum_{\al,\be}\frac{Q^{(\al)}(\x_n)}{\til{P}(\x_n)}\frac{\cc{Q^{(\be)}(\x_n)}}{\til{P}(\x_n)}
\frac1{\al!\be!}\int D^\al u\cc{D^\be u}\e^{\i\ska{x}{\x_n-\x_m}}\df x\right\}.
\end{equation}
Um $\de_{nm}\to0$ zu erreichen wählen wir die Folge $\x_n$ (z.B.~durch Übergang zu einer Teilfolge) so,
dass $\x_n-\x_m\to\infi$, für $n,m\to\infi$, $n\neq m$.
Der Faktor $D^\al u\cc{D^\be u}$ im Integral ist integrierbar.
Da auch die Faktoren vor dem Integral nach Voraussetzung beschränkt sind,
folgt $\de_{nm}\to0$ mit dem Riemann-Lebesgue-Lemma.

Aufgrund von \eqref{eq:2:PDunb} können wir zu einer Teilfolge übergehen,
so dass $Q(D)u_{n^\pr}$ konvergiert.
Dann gilt wegen \eqref{eq:2:kz} auch
\begin{equation}
\norm{Q(D)u_{n^\pr}}^2=\sum_{\al,\be}\frac{\abs{Q(\x_{n^\pr})}^2}{\til{P}(\x_{n^\pr})^2}\Psi_{\al\be}\to0.
\end{equation}
Daraus folgt auch $\til{Q}(\x_{n^\pr})^2/\til{P}(\x_{n^\pr})^2\to0$.
\end{proof}
