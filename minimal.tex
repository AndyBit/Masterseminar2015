% !TEX root = main.tex
\chapter{Minimale Operatoren}

Als erste Etappe in diesem Abschnitt,
führen wir den Vergleich der Stärke (Definition \ref{df:1:1.2})
zweier Differentialperatoren mit konstanten Koeffizienten,
auf einen Vergleich der Symbole zurück.

\section{Die Stärke von Differentialausdrücken}

Es seien in diesem Abschnitt stets $\mcP=P(\D)$,
$\mcQ=Q(\D)$ und $\Om$ ein beschränktes Gebiet.
Ziel ist die Charaktierisierung der Aussage ``$\mathcal Q$ ist schächer als $\mcP$''
durch eine Ungleichung der Form
\begin{equation}\label{eq:2:1}
\sup_{\x\in\R^n}\frac{\til{Q}(\x)}{\til{P}(\x)}<\infi,
\end{equation}
wobei $\til{P}$ und $\til{Q}$ geeignete Regularisierungen von $P$ und $Q$ sind.
Beachte, dass die reellen Nullstellen von $P$ Probleme machen würden,
wenn man $\til{P}=P$ verwendete.

In den folgenden Betrachtungen benötigt man häufig die Leibniz'sche Formel
\begin{equation}\label{eq:2:leib}
P(\D)(uv)=\sum_{\al}(P^{(\al)}(\D)v)\left(\frac{\D^\al u}{\al!}\right),
\quad\text{f.a.}~u,v\in C^\infi,
\end{equation}
für Differentialausdrücke.
Für den Differentialausdruck $P(\D)=\D^\al$
ist dies die gewöhnliche Leibnizregel
und somit folgt \eqref{eq:2:leib} aus Linearitätsgründen.
Dabei ist $P^{(\al)}(\D)$ der Differentialausdruck,
der zum Symbol
\begin{equation}
P^{(\al)}(\x)=\frac{\pd^{\abs{\al}}}{\pd\x_\al}P(\x),
\end{equation}
korrespondiert.
Wir treffen folgende Konvention:
Im Falle griechischer Indizes
ist die Ableitung nach einem Multiindex $\al\in\N^n_0$ gemeint,
im Fall latainischer Indizes oder Zahlen definieren wir
\begin{equation}
P^{(k_1,\dots,k_l)}(\x)=\frac{\pd}{\pd\x_{k_1}}\dots\frac{\pd}{\pd\x_{k_l}}P(\x).
\end{equation}

Im eindimensionalen Fall ist bekannt, dass $Q(\D)$ genau dann schwächer wie $P(\D)$ ist,
wenn $P(\z)$ höheren oder gleichen Grad wie $Q(\z)$ hat.
Die Resultate in diesem Abschnitt zeigen dies im Nachhinein.
%Zum Beweis genügt die mehrfache Anwendung des Theorems \ref{thm:1:1.8}
%auf den Operator $D$ und der Dreiecksungleichung.
Als Regularisierungen könnte man $\til{P}(\z)=1+\abs{P(\z)}^2$
und $\til{Q}(\z)=1+\abs{Q(\z)}^2$ verwenden.
Daß dies schon im Zweidimensionalen nicht mehr funktioniert,
hängt damit zusammen, dass $\abs{P^{(\al)}(\z)}^2/(1+\abs{P(\z)}^2)$
nun im allgemeinen nicht mehr beschränkt ist.
Man betrachte zum Beispiel $P(\z)=\z_1\z_2$ mit $P^{(1)}(\z)=\z_2$.

Es zeigt sich, dass die Regularisierungen
\begin{equation}\label{eq:2:2}
\til{P}(\z):=\sum_\al\abs{P^{(\al)}(\z)}^2,
\quad\text{f.a.}~\z\in\C^n,
\end{equation}
eine natürliche Wahl sind.
Also das Betragsquadrat des Vektors $(P^{(\al)}(\z))_\al\in\C^M$.
Beachte, dass für $\x\in\R^n$ der Ausdruck
$\til{P}(\x)$ ein reelles Polynom ohne Nullstellen ist.

\begin{thm}\label{thm:2:2.1}
Der Differentialoperator $\mcQ$ ist genau dann schwächer wie $\mcP$,
wenn
\begin{equation}\label{eq:thm:2:2.1}
\sup_{\x\in\R^n}\frac{\til{Q}(\x)}{\til{P}(\x)}<\infi
\end{equation}
erfüllt ist.
\end{thm}

Wir zeigen zunächst, wie in beiden Richtungen abgeschätzt wird.
Für die Rückrichtung benötigen wir ein Lemma,
dass wir später beweisen.

\begin{proof}
{\em Hinrichtung}:
Sei $\mcQ$ schwächer wie $\mcP$.
Dann gilt für ein $C\in(0,\infi)$ die Ungleichung
\begin{equation}\label{eq:2:4}
\norm{\mcQ u}\leq C(\norm{\mcP u}^2+\norm{u}^2),\quad\text{f.a.}~u\in C^\infi_0(\Om).
\end{equation}
Für jedes $\x\in\R^n$ und ein festes $\psi\in C^\infi_0(\Om)$, $\psi\neq0$,
definieren wir $\psi_\x(x):=\psi(x)\e^{\ska{x}{\x}}$, f.a.~$x\in\R^n$.
Dies bringt durch die Formel von Leibniz die Ableitungen von $P(\x)$ ins Spiel
\begin{equation}\label{eq:2.5}
P(\D)\psi_\x(x)=\e^{\ska{x}{\x}}\sum_\al P^{(\al)}(\x)\frac{\D^\alpha\psi(x)}{\al!},
\end{equation}
analoges gilt für $Q(\D)$.
Schreiben wir kurz
\begin{equation}\label{eq:2.6}
\Psi_{\al\be}:=\frac1{\al!\be!}\spro{\D^\alpha\psi}{\D^\beta\psi},
\end{equation}
so liest sich die Ungleichung \eqref{eq:2:4} mit $u=\psi_\x$ als
\begin{equation}\label{eq:2.7}
\sum_{\al,\be}Q^{(\al)}(\x)\bar{Q}^{(\be)}(\x)\Psi_{\al\be}
\leq C\left(\sum_{\al,\be}P^{(\al)}(\x)\bar{P}^{(\be)}(\x)\Psi_{\al\be}+\Psi_{00}\right).
\end{equation}
Da $\psi_\x\in C^\infi_0$, $\psi_\x\neq0$ gilt, sind $\norm{\mcP\psi_\x}$ und $\norm{\mcQ\psi_\x}$ stets positiv.
Somit ist $\left(\Psi_{\al\be}\right)_{\al,\be}$ auch die gram'sche Matrix
eines Skalarproduktes auf einem $\C^M$.
Es gibt also $\Psi_-,\Psi_+\in(0,\infi)$, so dass
\begin{equation}\label{eq:2.8}
\abs{\z}^2\Psi_-\leq\sum_{\al,\be}\z_\al\z_\be\Psi_{\al\be}\leq\abs{\z}^2\Psi_+,
\quad\text{f.a.}~\z\in\C^M.
\end{equation}
Die Ungleichungen \eqref{eq:2.7} und \eqref{eq:2.8} kombinieren sich zu
\begin{equation}\label{eq:2.9}
\til{Q}(\x)\Psi_-C\leq\Psi_+\til{P}(\x),
\quad\text{f.a.}~\x\in\R^n,
\end{equation}
und dies ist gerade die Ungleichung \eqref{eq:thm:2:2.1}.

{\em Rückrichtung}:
Sei die Ungleichung \eqref{eq:thm:2:2.1} erfüllt, d.h.
\begin{equation}\label{eq:2.10}
\til{Q}(\x)\leq C\til{P}(\x),
\quad\text{f.a.}~\x\in\R^n,
\end{equation}
mit $C\in(0,\infi)$.
Wir schätzen wie folgt ab
\begin{equation}\label{eq:2.11}
\norm{Q(\D)u}^2=\norm{Q\cdot\hat{u}}^2\leq C\norm{\sqrt{\til{P}}\cdot\hat{u}}^2
\leq C\sum_{\al}\norm{P^{(\al)}\cdot\hat{u}}^2=C\sum_\al\norm{P^{(\al)}(\D)u}^2.
\end{equation}
Um den Beweis abzuschließen,
fehlte also nur noch eine Ungleichung der Art
\begin{equation}
\norm{P^{(\al)}(\D)u}\leq C\norm{P(\D)u},
\quad\text{f.a.}~u\in C^\infi_0(\Om),~\al\in\N^n_0,
\end{equation}
mit $C\in(0,\infi)$.
Diese werden wir mit der Methode der Energieintegrale herleiten (Korollar \ref{cor:2:2.5}).
\end{proof}

\begin{rem}
Der Beweis zeigt,
dass in Formel \eqref{eq:thm:2:2.1} $\til{Q}(\x)$
durch $Q^2(\x)$ ersetzt werden kann.
\end{rem}

\section{Energieintegrale}

Da wir eine Methode brauchen,
um Skalarprodukte mit verschiedenen Differentialoperatoren abzuschätzen,
macht es Sinn, quadratische Differentialausdrücke $F$, definiert durch
\begin{equation}
F(\D,\overline{\D})u\bar{u}
:=\sum_{\al,\be}a_{\al\be}(\D^\alpha u)(\cc{\D^\beta u}),
\quad\text{f.a.}~u\in C^\infi_0(\Om),
\end{equation}
mit $a_{\al\be}\in\C$, genauer zu betrachten.
Uns interessieren die Terme
\begin{equation}\label{eq:2:qf}
\int F(\D,\overline{\D})u(x)\bar{u}(x)\d x
=\int F(\x,\x)\abs{\hat{u}(\x)}^2\d \x,
\quad\text{f.a.}~u\in C^\infi_0(\Om),
\end{equation}
die quadratische Formen auf $C^\infi_0(\Om)\ti C^\infi_0(\Om)\subs L^2(\R^n)\ti L^2(\R^n)$
definieren.
Es besteht eine eins-zu-eins korrespondenz
zwischen den quadratischen Differentialausdrücken $F(\D,\cc{\D})$
und den Symbolen $F(\x,\cc{\x})$, $\x\in\C^n$,
analog zu den Symbolen von Differentialausdrücken.

Die quadratischen Formen \eqref{eq:2:qf}
sind aber nur von den reellen Werten $F(\x,\x)$, $\x\in\R^n$ abhängig.
Verschiedene quadratische Differentialausdrücke bzw.~Symbole
können also die gleiche quadratische Form definieren.
Folgendes Lemma zeigt, dass es eine eins-zu-eins
Korrespendenz zwischen denen auf reelle Werte eingeschränkten
Symbole $F(\x,\x)$, $\x\in\R^n$,
und den quadratischen Formen wie in \eqref{eq:2:qf} gibt.

\begin{lem}\label{lem:2:qfsym}
Sei $\Om$ ein beliebiges Gebiet.
Die Gleichung
\begin{equation}\label{eq:lem:2:qfsym}
\int F(\D,\overline{\D})u\bar{u}\d x=0,
\quad\text{f.a.}~u\in C^\infi_0(\Om),
\end{equation}
gilt genau dann, wenn $F(\x,\x)=0$, f.a.~$\x\in\R^n$.
\end{lem}
\begin{proof}
Es gelte \eqref{eq:lem:2:qfsym}.
Sei $u_\y(x):=u(x)\e^{\ska{x}{\y}}$ für ein festes $u\in C^\infi_0(\Om)$, $u\neq0$.
Dann ist
\begin{equation}
\int F(\D,\overline{\D})u_\y\bar{u_\y}\d x
=\int F(\x,\x)\abs{\hat{u}(\x-\y)}^2\d \y
=\int F(\x+\y,\x+\y)\abs{\hat{u}(\x)}^2\d \x=:q(\y),
\end{equation}
für alle $\y\in\R^n$.
Auf der rechten Seite steht wieder ein Polynom $q(\y)$ in $\y\in\R^n$
und nach Voraussetzung müssen dessen Koeffizienten verschwinden.
Beachten wir, dass
\begin{equation}
(\x+\y)^\al=\sum_{\be+\ga=\al}\begin{pmatrix}\al\\\be\end{pmatrix}\x^\be\y^\ga=\y^\al+\dots
\end{equation}
so sehen wir, dass die führenden Terme von $q(\y)$
bis auf den Faktor $\norm{u}^2\neq0$ den führenden Termen von $F(\x,\x)$ entsprechen.
Wäre also $F(\x,\x)\neq0$, so hätten wir einen Widerspruch.

Die Rückrichtung sahen wir bereits, siehe Gleichung \eqref{eq:2:qf}.
\end{proof}

Unser Ziel ist es, Ableitungen von Ableitungspolynomen zu kontrollieren.
Dazu betrachten wir Ableitungen quadratischer Differentialformen.
Wir beginnen mit dem Ausdruck
\begin{equation}
\frac{\pd}{\partial x_k}
\left[F(\D,\overline{\D})u\bar{u}\right]
=\i\left[(\D_k-\overline{\D}_k)F(\D,\overline{\D})\right]u\bar{u}.
\end{equation}
Für einen Vektor $\ul{G}=(G_k)_{k=1,\dots,n}$,
bestehend aus quadratischen Differentialformen $G_k$,
erhält man für die Divergenz die Formel
\begin{subequations}
\begin{equation}\label{eq:2.15a}
\div(\ul{G}(\D,\overline{\D})u\bar{u})=\sum_{k=1}^n\frac{\pd}{\partial x_k}\left[G_k(\D,\overline{\D})u\bar{u}\right]=F(\D,\overline{\D})u\bar{u},
\end{equation}
wobei
\begin{equation}\label{eq:2.15b}
F(\z,\bar{\z})=\i\sum_{k=1}^n(\z_k-\bar{\z}_k)G_k(\z,\bar{\z})=-2\sum_{k=1}^n\y_kG_k(\z,\bar{\z}),
\end{equation}
\end{subequations}
wenn wir $\z=\x+\i\y$ schreiben.
\begin{lem}\label{lem:2:2.2}
Ein Polynom $F(\z,\bar{\z})$ in $\z=\x+\i\y$ und $\bar{\z}=\x-\i\y$
kann genau dann in der Form \eqref{eq:2.15b} dargestellt werden,
wenn $F(\x,\x)=0$ für alle $\x=0$.

In diesem Fall gilt
\begin{equation}\label{eq:2.16}
G_k(\x,\x)=-\tfrac12\left.\frac{\partial F(\x+\i\y,\x-\i\y)}{\partial \y_k}\right\rvert_{\y=0}.
\end{equation}
\end{lem}
\begin{proof}
Die Hinrichtung ist offensichtlich.
Die Rückrichtung folgt aus der (reellen) Taylorentwicklung von $F(\x+\i\y,\x-\i\y)$ in $(\x,\y)\in\R^{2n}$.
Der Satz von Taylor angewendet auf variables $\y\in\R^n$ und festes $\x\in\R^n$ liefert schließlich \eqref{eq:2.16}.
\end{proof}

Die Polynome $G_k(\z,\bar{\z})$, $\z\in\C^n$, sind als ganzes sind nicht eindeutig festgelegt.
Doch aufgrund von Lemma \ref{lem:2:qfsym} stört das nicht.

Mit Hilfe einer speziellen quadratischen Differentialformen und obiger Formel
gewinnen wir folgende Ungleichung,
mit der wir die Ableitungen der Differentialoperatoren kontrollieren können.

\begin{lem}
Sei $B_k:=\sup_{x_k,y_k\in\Om}\abs{x_k-y_k}<\infi$.
Für Ableitungspolynome $P(\D)$, $Q(\D)$ gilt die Ungleichung
\begin{equation}\label{eq:lem:2:2.4}
\abs{\spro{P^{(k)}(\D)u}{\bar{Q}(\D)u}}\leq\norm{P(\D)u}\left(\norm{\bar{Q}^{(k)}(\D)u}+B_k\norm{\bar{Q}(\D)u}\right),
\quad\text{f.a.}~u\in C^\infi_0(\Om).
\end{equation}
\end{lem}

\begin{proof}
Wir betrachten die quadratische Differentialform mit dem Symbol
\begin{equation}
F(\z,\bar{\z}):=P(\z)Q(\bar{\z})-Q(\z)P(\bar{\z}).
\end{equation}
Diese erfüllt offensichtlich $F(\x,\x)=0$, f.a.~$\x\in\R^n$.
Nach Lemma \ref{lem:2:2.2} (Formel \eqref{eq:2.16})
erhalten wir, nachdem wir mit $-\i x_k$ multipliziert haben
\begin{equation}\label{eq:2:2.22}
-\i x_kF(\D,\overline{\D})u\bar{u}=-\i x_k\sum_{k=1}^n\frac{\pd}{\partial x_k}\left[G_k(\D,\overline{\D})u\bar{u}\right],
\end{equation}
mit
\begin{equation}
G_k(\x,\x)=-\i\left(P^{(k)}(\x)Q(\x)-Q^{(k)}(\x)P(\x)\right),
\quad\text{f.a.}~\x\in\R^n.
\end{equation}

Integrieren wir die Gleichung \eqref{eq:2:2.22},
so liefert eine partielle Integration auf rechten Seite den Term
\begin{align}
\i\int G_k(\D,\overline{\D})u\bar{u}\d x&=\int\left(P^{(k)}(\D)Q(\bar{D})-P(\D)Q^{(k)}(\D)\right)u\bar{u}\d x\\
&=\spro{P^{(k)}(\D)u}{\bar{Q}(\D)u}-\spro{P(\D)u}{\bar{Q}^{(k)}(\D)u}.
\end{align}
Damit erhalten wir insgesamt
\begin{equation}
\spro{P^{(k)}(\D)u}{\bar{Q}(\D)u}
=\spro{P(\D)u}{\bar{Q}^{(k)}(\D)u}-i\int x_k\left(P(\D)u\cc{\bar{Q}(\D)u}-Q(\D)u\cc{\bar{P}(\D)u}\right)\d x.
\end{equation}
Ohne Beschränkung der Allgemeinheit können wir das Gebiet $\Om$ so legen,
dass $\abs{x_k}\leq B_k/2$ für alle $x_k\in\Om$.
Man beachte, dass Differentialausdrücke mit Translationsoperatoren vertauschen.
Schließlich erhalten wir mit Cauchy-Schwarz
und der Dreiecksungleichung die Ungleichung \eqref{eq:lem:2:2.4}.
\end{proof}

\begin{cor}\label{cor:2:2.5}
Ist $P(\x)$ vom Grad $m$ in $\x_k$, so gilt
\begin{equation}\label{eq:cor:2:2.5}
\norm{P^{(k)}(\D)u}\leq mB_k\norm{P(\D)u},
\quad\text{f.a.}~u\in C^\infi_0(\Om).
\end{equation}
\end{cor}
\begin{proof}
Setzen wir $\bar{Q}(\x)=P^{(k)}(\x)$, so erhalten wir aus \eqref{eq:lem:2:2.4} die Ungleichung
\begin{equation}\label{eq:2.28}
\norm{P^{(k)}(\D)u}^2\leq\norm{P(\D)u}\left(\norm{P^{(kk)}(\D)u}+B_k\norm{P^{(k)}(\D)u}\right),
\quad\text{f.a.}~u\in C^\infi_0(\Om).
\end{equation}
Wir gehen nun Induktiv in $m$ vor.
Für $m=1$ ist $P^{(kk)}=0$, $P^{(k)}\neq0$ und somit entspricht die Gleichung \eqref{eq:cor:2:2.5}
gerade der Gleichung \eqref{eq:lem:2:2.4} nach kürzen eines Faktors.
Angenommen das Korollar ist für $m-1$ erfüllt,
dann erhalten wir durch Kombination von \eqref{eq:2.28} und \eqref{eq:cor:2:2.5}
\begin{equation}
\norm{P^{(k)}(\D)u}^2\leq\norm{P(\D)u}\left((m-1)B_k\norm{P^{(k)}(\D)u}+B_k\norm{P^{(k)}(\D)u}\right),
\end{equation}
also ist auch der Induktionsschritt gezeigt.
\end{proof}

\begin{cor}\label{cor:2:2.6}
Für festes beschränktes Gebiet $\Om$ und beliebiges $\al\in\N^n_0$ gilt
\begin{equation}
\norm{P^{(\al)}(\D)u}\leq C_{\al,\Om}\norm{P(\D)u},
\quad\text{f.a.}~u\in C^\infi_0(\Om),
\end{equation}
mit $C_{\al,\Om}=m(m-1)\cdots(m-\abs{\al})B^\al$, wobei $B^\al=\prod_{l=1}^n B^{\al_l}_l$.
Die Konstante $C_{\al,\Om}$ hängt also nur vom Grad $m$ von $P(\x)$,
der Ableitungsordnung $\abs{\al}$ und der Ausdehnung des Gebiets $\Om$ ab.
\end{cor}

\begin{proof}
Mehrfache Anwendung von Korollar \ref{cor:2:2.5}.
\end{proof}

Das letzte Korollar ist gerade der im Beweis von Satz \ref{thm:2:2.1}
gebrauchte Baustein.

\section{Beispiele und spezielle Symbole}

Wir betrachten als instruktive Beispiele
einige klassische Differentialausdrücke zweiter Ordnung.

\begin{exa}\label{exa:2:lap}
Wir betrachten das zum Laplace-Operator $\Lap$
korrespondierende Symbol $P(\x)=\x_1^2+\dots+\x_n^2=\abs{\x}^2$.
Das regularisierte Symbol ist gegeben durch
\begin{equation}
\til{P}(\x)=\abs{\x}^4+4\abs{\x}^2+4n.
\end{equation}
Beachten wir, dass stets
\begin{equation}\label{eq:2:sqrest}
a^2+b^2\leq(a+b)^2\leq2(a^2+b^2),\quad\text{für}~a,b\geq0,
\end{equation}
so sehen wir, dass
\begin{equation}
\til{P}(\x)\asymp1+\abs{\x}^4.
\end{equation}
Also ist der Laplaceoperator stärker wie alle Differentialoperatoren
mit Symbolen von gleichem oder kleinerem Grad.
Diese sind alle von der Form
\begin{equation}
Q(\x)=\x^\tr A\x+\ska{b}{\x}+c,
\end{equation}
mit einer Matrix $A\in\C^{n\ti n}$, $b\in\C^n$ und $c\in\C$,
wobei wir $A^\tr=A$ annehmen können.
Wir untersuchen, welche davon gleich stark wie der Laplaceoperator sind.
Es ist
\begin{equation}\label{eq:2:rpQLap1}
\til{Q}(\x)=\abs{\x^\tr A\x+\ska{b}{\x}+c}^2+\sum_{k=1}^n\abs{2e^\tr_kA\x+b_k}^2
+4\sum_{1\leq k\leq l\leq n}\abs{a_{kl}}^2.
\end{equation}
Da die Ausdrücke $\til{Q}(\x)$ und $1+\abs{\x^\tr A\x}^2$
auf kompakten Mengen positiv und beschränkt sind (falls $\mcQ\neq0$) genügt es,
diese für $\abs{\x}\geq R$ mit festem $R>0$ groß genug zu betrachten.
Ist $\abs{\x^\tr A\x}\neq0$ falls $\x\neq0$,
so gilt $\abs{\x^\tr A\x}\asymp\abs{\x}^2$.
In diesem Fall kann man, für $R$ groß genug,
die linearen Terme in \eqref{eq:2:rpQLap1} vernachlässigen
und erhält insgesamt
\begin{equation}
\til{Q}(\x)\asymp1+\abs{\x^\tr A\x}^2\asymp1+\abs{\x}^4\asymp\til{P}(\x).
\end{equation}
Im anderen Fall betrachtet man \eqref{eq:2:rpQLap1}
für alle $\x\in\R^n$ mit $\x^\tr A\x=0$,
und stellt fest, dass $\til{Q}(\x)$ für diese $\x$ nur linear wächst,
$Q(\D)$ also echt schwächer wie $P(\D)$ ist.

Im Fall einer reellen Matrix, ist $\x^\tr A\x\neq0$ für $\x\neq0$,
falls entweder alle Eigenwerte positiv oder alle negativ sind.
Im Fall einer komplexen Matrix ist
\begin{equation}
\abs{\x^\tr A\x}^2=\abs{\x^\tr A_\Re\x}^2+\abs{\x^\tr A_\Im\x}^2,
\end{equation}
mit $A=A_\Re+\i A_\Im$, $A_\Re,A_\Im\in\R^{n\ti n}$.
Die weitere Untersuchung gestaltet sich allerdings recht aufwendig
und soll nicht weiter vertieft werden.
\end{exa}

\begin{exa}\label{exa:2:schroe}
Der Schrödingergleichung für ein freies Teilchen
\begin{equation}
\i\frac{\pd}{\partial t}\psi(x,t)=-\Lap\psi(x,t),\end{equation}
lässt sich das Symbol $P(\x)=\x^2_1+\dots+\x^2_{n-1}-\x_n$ zuordnen.

Wir bestimmen alle Operatoren gleicher Stärke.
Das Symbol $Q(\x)$ ist genau dann schwächer wie $P(\x)$,
wenn
\begin{equation}\label{eq:2:subschroe}
Q(\x)^2\apprle(\x^2_1+\dots+\x^2_{n-1}-\x_n)^2+\x^2_1+\dots+\x^2_{n-1}+1.
\end{equation}
Also hat $Q(\x)$ maximalen Grad 2 in $\x_1,\dots,\x_{n-1}$,
Grad 1 in $\x_n$ und ist somit von der Form
\begin{equation}
Q(\x)=a_0+\sum^n_{k=1}a_k\x_k+\sum^n_{k,l=1}a_{kl}\x_k\x_l,
\end{equation}
mit $a_{nn}=0$.
Wir betrachten die Gleichung \eqref{eq:2:subschroe}
für spezielle Vektoren:
\begin{equation}\label{eq:2:subschroespec}
Q(\x)^2\apprle(\x^2_1+\dots+\x^2_{n-1}+1),
\quad\text{für alle}~\x\in\R^n~\text{mit}~\x_n=\x^2_1+\dots+\x^2_{n-1}.
\end{equation}
Da wir $\x_1,\dots,\x_{n-1}$ in \eqref{eq:2:subschroespec} frei wählen können folgt, dass
\begin{equation}
Q(\x)=a_0+\sum_{k=1}^{n-1}a_k\x_k+a_n(\x_n-\x^2_1-\dots-\x^2_{n-1}),
\end{equation}
mit beliebigen $a_0,a_1,\dots,a_n\in\C$.
Es ist $Q(\x)$ genau dann gleich Stark wie $P(\x)$,
wenn $a_n\neq0$.
\end{exa}

\begin{exa}\label{exa:2:heat}
Der Wärmeleitungsgleichung
\begin{equation}
\frac{\pd}{\pd t}T(x,t)=\Lap T(x,t),
\end{equation}
entspricht das Symbol $P(\x)=\x^2_1+\dots+\x^2_{n-1}+\i\x_n$.

Ein Symbol $Q(\x)$ ist genau dann schwächer wie $P(\x)$, wenn
\begin{equation}
Q(\x)^2\apprle(\x^2_1+\dots+\x^2_{n-1})^2+\x^2_1+\dots+\x^2_{n-1}+\x^2_n+1.
\end{equation}
Diese Ungleichung ist genau dann erfüllt, wenn
\begin{equation}
Q(\x)=a_0+\sum_{k=1}^na_k\x_k+\sum_{k,l=1}^{n-1}a_{kl}\x_k\x_l,
\end{equation}
mit beliebigen $a_0,a_k,a_{kl}\in\C$.
Das Symbol $Q(\x)$ ist genau dann gleich stark wie $P(\x)$,
wenn $a_n\neq0$ und $\sum_{k,l=1}^{n-1}a_{kl}\x_k\x_l\neq0$ falls $\x\notin e_n\R$
mit dem $n$-ten Standartbasisvektor $e_n$.
\end{exa}

\begin{exa}\label{exa:2:hyper}
Als letztes Beispiel betrachten wir die Gleichung
\begin{equation}
\Lap_1u=\Lap_2u,
\end{equation}
mit $\Lap_1:=\pd_1^2+\dots+\pd_m^2$, $\Lap_2:=\pd^2_{m+1}+\dots+\pd^2_m$.
Das korrespondierende Symbol ist $P(\x)=\x^2_1+\dots+\x^2_m-\x^2_{m+1}-\dots-\x^2_n$.

Ein Symbol $Q(\x)$ ist genau dann gleich stark wie $P(\x)$,
wenn
\begin{equation}
Q(\x)^2\apprle(\x^2_1+\dots+\x^2_m-\x^2_{m+1}+\dots+\x^2_n)^2+\x^2_1+\dots+\x^2_n+1.
\end{equation}
Setzen wir wie in Beispiel \ref{exa:2:schroe} $\x^2_1+\dots+\x^2_m=\x^2_{m+1}+\dots+\x^2_n$,
so erhalten wir für diese $\x$:
\begin{equation}\label{eq:2:rpQhyp}
Q(\x)^2\apprle\x^2_1+\dots+\x^2_n+1.
\end{equation}
Da wir in \eqref{eq:2:rpQhyp} noch genug Wahlfreiheit haben folgt,
dass $Q(\x)$ von der Form
\begin{equation}
Q(\x)=a_0+\sum_{k=1}^na_k\x^k+b(\x_1^2+\dots+\x^2_m-\x^2_{m+1}+\dots+\x^2_n),
\end{equation}
mit beliebigen $a_0,a_1,\dots,a_n,b\in\C$ sein muss
und $Q(\x)$ ist genau dann gleich stark wie $P(\x)$,
wenn $b\neq0$.
\end{exa}

Wir wollen einige Unterschiede der betrachteten Beispiele festhalten.
Vergleichen wir die Beispiele \ref{exa:2:lap}, \ref{exa:2:hyper}
mit den Beispielen \ref{exa:2:schroe}, \ref{exa:2:heat}.
In den ersten beiden Fällen hängt der Vergleich der Stärke nur von den Hauptteilen,
d.h.~den Potenzen größter Ordnung ab.
In den zweiten zwei Fällen hängt die Vergleich auch von niederen Termen ab.
Dies hängt damit zusammen, dass die Symbole $P(\x)$ in
\ref{exa:2:schroe} und \ref{exa:2:heat} nicht auf allen (nicht-trivialen)
Unterräumen von $\R^n$ wie die höchste im Symbol vorkommende Potenz wachsen.

Auch die Paare \ref{exa:2:lap}, \ref{exa:2:heat}
und \ref{exa:2:schroe}, \ref{exa:2:hyper}
weisen einen krassen Unterschied auf.
Bei den ersten beiden Operatoren hat man relativ viel Wahlfreiheit
für gleich Starke Operatoren.
Bei den zweiten zwei Operatoren hat man für den Hauptteil stets
nur einen freien Parameter.
Dies hängt damit zusammen, dass in den Symbolen $P(\x)$
in \ref{exa:2:schroe} und \ref{exa:2:hyper} Differenzterme auftreten.

Für besonders hervorstechende Vergleichbarkeitseigenschaften
wurden folgende Operatortypen eingeführt.

\begin{df}
\begin{enumerate}
\item
Ein Differentialausdruck $P(\D)$ heißt vom {\em Haupttyp},
falls er gleich stark ist wie jeder Differentialoperator mit gleichem Hauptteil.
\item
Ein Differentialausdruck $P(\D)$ heißt {\em elliptisch},
falls er stärker ist wie jeder Differentialausdruck von kleiner oder gleichem Grad.
\end{enumerate}
\end{df}

Wie eben diskutiert sind also der Laplaceoperator
und der Operator aus Beispiel \ref{exa:2:hyper} vom Haupttyp,
die anderen beiden nicht.
Der Laplaceoperator ist als einziger elliptisch.

\section{Weitere Vergleichsresultate}

Sehr analog zu Satz \ref{thm:2:2.1} lassen sich die regularisierten
Symbole $\til{P}$, $\til{Q}$ heranziehen
um die Kompaktheit der Abbildung
\begin{equation}\label{eq:2:map}
\mcR_{P_0}\to\mcR_{Q_0},\quad P(\D)u\mto Q(\D)u,
\end{equation}
zu charakterisieren (falls sie existiert).
Existiert die Abbildung \eqref{eq:2:map} und ist kompakt,
so nennen wir $Q(\D)$ relativ kompakt zu $P(\D)$.
Oben hatten wir charakterisiert,
wann die Abbildung \eqref{eq:2:map} existiert und stetig ist.

\begin{thm}
Der Operator $Q(\D)$ ist genau dann relativ kompakt zu $P(\D)$,
wenn
\begin{equation}\label{eq:2:tozero}
\frac{\til{Q}(\x)}{\til{P}(\x)}\to0,\quad\text{für}\quad\x\to\infi.
\end{equation}
\end{thm}
\begin{proof}
Sei die Gleichung \eqref{eq:2:tozero} erfüllt.
Wir wählen eine beliebige Folge $u_n\in C^\infi_0(\Om)$,
so dass 
\begin{equation}\label{eq:2:PDb}
\norm{P(\D)u_n}\leq1.
\end{equation}
Wir zeigen, dass es eine Teilfolge $u_{n^\pr}$ gibt,
so dass $Q(\D)u_{n^\pr}$ konvergiert.
Zunächst stellen wir fest, dass der Ausdruck $\til{Q}(\x)/\til{P}(\x)$
stets lokal-beschränkt ist.
Somit impliziert \eqref{eq:2:tozero} bereits \eqref{eq:thm:2:2.1}
und nach Satz \ref{thm:2:2.1} ist damit
\begin{equation}
\norm{Q(\D)u_n}\leq C,
\quad\text{f.a.}~n\in\N,
\end{equation}
für ein $C\in(0,\infi)$.
Nach Hölder und der Stetigkeit der Fouriertransformation
von $L^1$ nach $L^\infi$ impliziert dies
\begin{equation}
\sqrt{2\pi}\norm{\D^\alpha(Q(\x)u_n)}_\infi\leq\norm{x^\al Q(\D)u_n}_1\leq\norm{Q(\D)u_n}\norm{x^\al1_\Om}
\leq\norm{Q(\D)u_n}\sqrt{\abs{\Om}}\tfrac{B^\al}{2^{\abs{\al}}}.
\end{equation}
In Kombination mit dem Riemann-Lebesgue-Lemma folgt,
dass alle Ableitungen von $Q(\x)\hat{u}_n(\x)$
gleichmäßig beschränkt sind und im unendlichen abfallend.
Insbesondere lässt sich eine lokal-gleichmäßig konvergente Teilfolge $Q(\x)\hat{u}_{n^\pr}(\x)$ auswählen.

Wir zeigen, dass dies unser Kandidat ist.
Nach Voraussetzung lässt sich zu jedem $\ep>0$ eine kompakte Menge $K$ wählen,
so dass $\abs{Q(\x)}/\til{P}(\x)<\ep$ für $\x$ im Komplement $K^{\mathrm c}$ von $K$.
Nun ist
\begin{align}
\int_{K^{\mathrm c}}\abs{Q(\x)}^2\abs{\hat{u}_{n^\pr}(\x)-\hat{u}_{m^\pr}(\x)}^2\d \x,
&\leq\ep^2\int\til{P}(\x)\abs{\hat{u}_{n^\pr}(\x)-\hat{u}_{m^\pr}(\x)}^2\d \x\\
&=\ep^2\sum_{\al\in\N^n_0}\norm{P^{(\al)}(\D)(u_{n^\pr}-u_{m^\pr})}^2,\label{eq:2:Kcsum}
\end{align}
und die Summe in \eqref{eq:2:Kcsum} ist nach Korollar \ref{cor:2:2.6}
und \eqref{eq:2:PDb} beschränkt.
Weiter gilt
\begin{equation}
\int_K\abs{Q(\x)}^2\abs{\hat{u}_{n^\pr}(\x)-\hat{u}_{m^\pr}(\x)}^2\d \x\to0,
\quad\text{falls}~n^\pr~\text{und}~m^\pr\to\infi,
\end{equation}
aufgrund der lokal-gleichmäßigen Konvergenz.
Da $\ep>0$ beliebig gewählt werden konnte,
impliziert dies die Konvergenz von $Q(\D)u_{n^\pr}$ in $L^2$.

Nehmen wir nun an $Q(\D)$ ist kompakt relativ zu $P(\D)$.
Dazu weisen wir nach, dass aus $\x_n\to\infi$ die Konvergenz $\til{Q}(\x_n)/\til{P}(\x_n)\to0$ folgt.
Wähle dazu $u\in C^\infi_0(\Om)$, $u\neq0$ fest und setze
\begin{equation}
u_n(x):=u(x)\frac{\e^{\i\ska{x}{\x_n}}}{\til{P}(\x_n)},\quad\text{f.a.}~u\in\R^n,
\end{equation}
für jedes $n\in\N$.
Wendet man die Leibnizsche Formel auf $u_n(x)$ an,
so erhält man
\begin{equation}
P(\D)u_n(x)=\e^{\i\ska{x}{\x_n}}\sum_\al\frac{P^{(\al)}(\x_n)}{\til{P}(\x_n)}\frac{\D^\alpha u(x)}{\al!}.
\end{equation}
Beachtet man, dass die Ausdrücke $P^{(\al)}/\til{P}$ stets beschränkt sind
und wendet wieder einmal Korollar \ref{cor:2:2.6} an,
so erhält man, dass
\begin{equation}\label{eq:2:PDunb}
\norm{P(\D)u_n}\leq C,
\end{equation}
für ein $C\in(0,\infi)$.
Wir zerlegen nun
\begin{equation}\label{eq:2:kz}
\norm{Q(\D)u_n-Q(\D)u_m}^2=\norm{Q(\D)u_n}^2+\norm{Q(\D)u_m}^2-\de_{nm},
\end{equation}
wobei
\begin{equation}
\de_{nm}=2\Re\left\{\sum_{\al,\be}\frac{Q^{(\al)}(\x_n)}{\til{P}(\x_n)}\frac{\cc{Q^{(\be)}(\x_n)}}{\til{P}(\x_n)}
\frac1{\al!\be!}\int \D^\alpha u\cc{\D^\beta u}\e^{\i\ska{x}{\x_n-\x_m}}\d x\right\}.
\end{equation}
Um $\de_{nm}\to0$ zu erreichen wählen wir die Folge $\x_n$ (z.B.~durch Übergang zu einer Teilfolge) so,
dass $\x_n-\x_m\to\infi$, für $n,m\to\infi$, $n\neq m$.
Der Faktor $\D^\alpha u\cc{\D^\beta u}$ im Integral ist integrierbar.
Da auch die Faktoren vor dem Integral nach Voraussetzung beschränkt sind,
folgt $\de_{nm}\to0$ mit dem Riemann-Lebesgue-Lemma.

Aufgrund von \eqref{eq:2:PDunb} können wir zu einer Teilfolge übergehen,
so dass $Q(\D)u_{n^\pr}$ konvergiert.
Dann gilt wegen \eqref{eq:2:kz} auch
\begin{equation}
\norm{Q(\D)u_{n^\pr}}^2=\sum_{\al,\be}\frac{\abs{Q(\x_{n^\pr})}^2}{\til{P}(\x_{n^\pr})}\Psi_{\al\be}\to0.
\end{equation}
Daraus folgt auch $\til{Q}(\x_{n^\pr})/\til{P}(\x_{n^\pr})\to0$.
\end{proof}
