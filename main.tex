\documentclass[11pt,bibtotoc]{scrbook}
\usepackage{german}
\usepackage[utf8]{inputenc}
%
\usepackage{hyperref}
%
\usepackage{wasysym}\let\iint\relax\let\iiint\relax
\usepackage{amsfonts,amssymb,amsthm,amsmath,CJK,mathrsfs}
\usepackage[a4paper, scale=0.76]{geometry}
%
\usepackage{ifthen}
\usepackage{makeidx}
\makeindex
\newcommand{\eIndex}[2][]{%
\emph{#2}%
\ifthenelse{\equal{#1}{}}{\index{#2}}{\index{#1!#2}}%	
}%
\def\spro#1#2{\pmb(#1,#2\pmb)}
%
\newtheorem{thm}{Satz}[chapter]
\newtheorem{lem}[thm]{Lemma}
\newtheorem{cor}[thm]{Korollar}
\renewcommand{\proofname}{Beweis}
\theoremstyle{definition}
\newtheorem{df}[thm]{Definition}
\newtheorem{exa}{Beispiel}[chapter]
\newtheorem{rem}[thm]{Bemerkung}
%
%
% DEFINITIONEN
\def\R{\mathbb R}    % Koerper R
\def\C{\mathbb C}    % Koerper C
\def\e{\mathrm e}     % Eulersche Zahl e
\def\i{\mathrm i}       % imaginäre Einheit i
\def\D{\mathrm D}    % Differentialoperator D
\def\d{\,\mathrm d}   % Differential d für Integrale
\def\rmC{\mathrm C}% Raum C
\def\rmL{\mathrm L} % Raum L
\def\rmH{\mathrm H}% Raum H
%% von Tillmann
\def\N{\mathbb N}
\def\mcP{\mathcal P}
\def\mcQ{\mathcal Q}
\def\mcR{\mathcal R}
\def\pr{\prime}
\def\tr{\top}%{\mathrm{T}}
\def\al{\alpha}
\def\be{\beta}
\def\de{\delta}
\def\ep{\epsilon}
\def\ga{\gamma}
\def\x{\xi}
\def\y{\eta}
\def\z{\zeta}
\def\Om{\Omega}
\def\pd{\partial}
\def\infi{\infty}
\def\til#1{\widetilde{#1}}
\def\mto{\mapsto}
\def\ti{\times}
\def\abs#1{\lvert#1\rvert}
\def\norm#1{\lVert#1\rVert}
\def\Lap{\Delta}
\def\cc#1{\overline{#1}}
\def\ul#1{\underline{#1}}
\def\ska#1#2{\langle#1,#2\rangle}
\def\Ska#1#2{\left\langle#1,#2\right\rangle}
\def\div{\operatorname{div}}
\def\diam{\operatorname{diam}}
\def\subs{\subset}
%%%%%%%%%%%%%%%%%%%%
%
\DeclareMathOperator{\supp}{supp}
%
\begin{document}
\titlehead{Priv.-Doz. Dr. Jens Wirth\\Institut f\"ur Analysis, Dynamik und Modellierung\\Universit\"at Stuttgart}
\lowertitleback{\copyright 2015. Texte geschrieben von den Teilnehmern des Seminars.  Einzelne Beiträge sind nicht namentlich gekennzeichnet. Ausarbeitungen basieren auf Originalliteratur. }
\title{Mikrolokale Analysis}
\subtitle{Masterseminar}
\author{---}
\date{Sommersemester 2015}
\maketitle
\tableofcontents
%
\part{Analysis von Differentialoperatoren}
%
% !TEX root = main.tex
\chapter{Einleitung}
Die folgenden Abschnitte basieren im wesentlichen auf Hörmanders Arbeit \cite{Hormander:1955}. In dieser Einleitung sollen die verwendete Notation festgelegt und die wichtigsten operatortheoretischen Konzepte zusammengefaßt werden.

\section{Funktionalanalytische Grundlagen}

Seien $V$ und $W$ Banachräume und $\mathcal D_T\subseteq V$ ein linearer Teilraum. Ein (im allgemeinen unbeschr\"ankter) \eIndex{Operator} $T$ mit Definitionsbereich $\mathcal D_T$ ist eine lineare Abbildung $T:V\supset\mathcal D_T \to W$. Er heißt \eIndex[Operator]{beschr\"ankt}, falls es eine Konstante $C>0$ mit 
\begin{equation}
\forall v\in\mathcal D_T \quad:\quad \|Tv \|_W \le C \|v\|_V
\end{equation}
gibt. Weiter heißt
\begin{equation}
\mathcal R_T = \{ Tv : v\in \mathcal D_T \} \subseteq W
\end{equation}
der \eIndex[Operator]{Wertebereich} und 
\begin{equation}
\mathcal G_T = \{ (v,Tv) : v \in \mathcal D_T \} \subseteq V\times W 
\end{equation}
der \eIndex[Operator]{Graph} von $T$. Eine Teilmenge $\mathcal G\subseteq V\times W$ ist genau dann Graph eines Operators, wenn $\mathcal G$ linear ist und aus
$(0,w)\in\mathcal G$ stets $w=0$ folgt. Wir versehen $V\times W$ mit der Norm $\|(v,w)\|_{V\times W}^2 = \|v\|_V^2 + \|w\|_W^2$.


Der Operator $T$ wird als \eIndex[Operator]{abgeschlossen} bezeichnet, falls der Graph $\mathcal G_T$ ein abgeschlossener Teilraum des Produktraumes $V\times W$ ist. Weiter heißt $T$ \eIndex[Operator]{abschließbar}, falls der Abschluß von $\mathcal G_T$ in $V\times W$ Graph eines Operators ist. Dieser wird als Abschluß von $T$ bezeichnet.

\begin{thm}\label{thm:1:1.1}
Sei $T_1$ abgeschlossen und $T_2$ abschließbar und gelte $\mathcal D_{T_1}\subseteq \mathcal D_{T_2}$. Dann existiert eine Konstante $C$ mit  
\begin{equation}
   \| T_2 u\|_W^2 \le C ( \|T_1 u\|_W^2 + \|u\|_V^2 ). 
\end{equation}
\end{thm}
\begin{proof}
Wir betrachten die Abbildung $\mathcal G_{T_1} \ni (u,T_1u) \mapsto T_2 u \in W$. Diese ist linear und überall definiert. Wir zeigen, daß sie auch abgeschlossen ist. Angenommen, $u_n\to u$ und $T_1 u_n\to T_1 u$ und $T_2 u_n$ sei konvergent. Da $T_2$ abschließbar ist, konvergiert $T_2 u_n \to T_2 u$. Also ist die Abbildung abgeschlossen, nach dem Satz vom abgeschlossenen Graphen stetig und somit beschr\"ankt.
\end{proof}

Ist $T$ injektiv, so bestimmt $\mathcal G_{T^{-1}} = \{ (w,v) : (v,w)\in\mathcal G_T\}$ einen Graphen. Der zugehörige Operator wird als $T^{-1}$ bezeichnet. Dieser ist genau dann abgeschlossen, wenn $T$ abgeschlossen ist. Ist $T^{-1}$ beschränkt, gilt also
\begin{equation}\label{eq:1:1.6} 
   \forall u\in\mathcal D_T \quad:\quad \|u\|_V\le C \|Tu\|_W
\end{equation}
mit einer von $u$ unabhängigen Konstanten $C$, so sagen wir $T$ ist \eIndex[Operator]{beschränkt invertierbar}.

Im folgenden sei $V=W=H$ ein Hilbertraum. Zur Vereinfachung der Notation verwenden wir immer $\|\cdot\|$ f\"ur die Norm in $H$. Weiter sei das Innenprodukt in $H$ durch $\spro{u}{v}$ bezeichnet und $H\times H$ mit dem entsprechenden Innenprodukt $\spro{(u_0,u_1)}{(v_0,v_1)} = \spro{u_0}{v_0} + \spro{u_1}{v_1}$ versehen. Auf $H\times H$ sei der Operator $J$ mit $J(v,w) = (-w,v)$ definiert. Dann ist zu jedem dicht definierten Operator $T$ mit Graphen $\mathcal G_T$ durch $\mathcal G_{T^*}=(J\mathcal G_T)^\perp$ ein Graph gegeben. Der zugeh\"orige Operator $T^*$ wird als zu $T$ \eIndex[Operator]{adjungiert} bezeichnet. Er ist immer abgeschlossen und es gilt
\begin{equation}
   \spro{Tu}{v} = \spro{u}{T^*v}
\end{equation}
f\"ur $u\in\mathcal D_T$ und $v\in\mathcal D_{T^*}$.

\begin{thm}\label{thm:1:1.2}
Sei $T$ ein dicht definierter Operator. Dann gilt $\mathcal R_T=H$ genau dann, wenn 
$T^*$ beschränkt invertierbar ist.
\end{thm}
\begin{proof}
Sei $\mathcal R_T=H$. Dann existiert zu jedem $u\in H$ ein $w\in H$ mit $Tw=u$. Damit folgt
\begin{equation}
   \spro{u}{v} = \spro{Tw}{v} = \spro{w}{T^*v}, \qquad |\spro{u}{v}| \le C_u \|T^*v\|
\end{equation}
f\"ur alle $u\in H$ und $v\in\mathcal D_{T^*}$ und mit dem Satz über die gleichmäßige Beschränktheit die Behauptung.

Angenommen, $T^*$ erf\"ullt die Ungleichung \eqref{eq:1:1.6}. Dann ist der selbstadjungierte Operator $TT^*$ wegen
\begin{equation}
  \spro{TT^*v}{v} = \spro{T^*v}{T^*v} \ge C^{-2} \spro{v}{v}
\end{equation}
strikt positiv und stetig invertierbar. Dann ist aber $TT^* (TT^*)^{-1}  = I$ und somit $\mathcal R_T = H$.
\end{proof}

\begin{cor}\label{cor:1:1.3}
Ein dicht definierter Operator $T$ besitzt eine Rechtsinverse $S$ genau dann, wenn $T^*$ beschränkt invertierbar ist.
\end{cor}
\begin{proof}
Wenn $T$ eine Rechtsinverse $S$ besitzt, impliziert $RS=I$ schon $\mathcal R_T=H$ und mit obigem Satz folgt die beschränkte Invertierbarkeit von $T^*$. Gilt umgekehrt \eqref{eq:1:1.6}, so ist $S=T^*(TT^*)^{-1}$ das gesuchte. Da $T^*(TT^*)^{-1/2}$ eine Isometrie ist, ist $S$ stetig.
\end{proof}

\section{Differentialoperatoren} 
Wir nutzen im folgenden Multiindexschreibweise. Für Multiindices $\alpha,\beta\in\mathbb N_0^n$ sei
\begin{equation}
|\alpha|=\sum_{j=1}^n \alpha_j,\qquad \alpha! = \prod_{j=1}^n \alpha_j!,\qquad (\alpha+\beta)_j=\alpha_j+\beta_j.
\end{equation} 
Weiter sei zu $\zeta\in\C^n$ durch
\begin{equation}
   \zeta^\alpha = \prod_{j=1}^n \zeta_j^{\alpha_j}
\end{equation}
das Monom, sowie durch
\begin{equation}
   \D^\alpha = \prod_{j=1}^n \left(- \i \frac{\partial}{\partial x_j}\right)^{\alpha_j}
\end{equation}
ein formaler Differentialoperator definiert. Sei nun $\Omega\subseteq\R^n$ ein Gebiet. Ein Differentialoperator der Ordnung $m$ ist ein formaler Ausdruck der Form
\begin{equation}
   \mathcal P = \sum_{|\alpha|\le m} a_\alpha(x) \D^\alpha
\end{equation}
mit Koeffizientenfunktionen $a_\alpha\in \rmC^\infty(\Omega)$. Er agiert in nat\"urlicher Weise auf $u\in\rmC^\infty_0(\Omega)$ (oder auf $u\in\mathscr D'(\Omega)$ in distributionellem Sinne).

Versieht man $\rmC_0^\infty(\Omega)$ durch
\begin{equation}
    \spro{u}{v} = \int_{\Omega} u(x) \overline{v(x)} \d x
\end{equation}
mit einem $\rmL^2$-Innenprodukt, so kann man zu $\mathcal P$ den formal adjungierten Operator $\overline{\mathcal P}$ betrachten. Dieser erf\"ullt
\begin{equation}
   \forall u,v\in\rmC_0^\infty(\Omega)\quad:\quad \spro{\mathcal Pu}{v}=\spro{u}{\mathcal{\overline P}v}
\end{equation}
und ist durch
\begin{equation}
  \overline{ \mathcal P}v(x) = \sum_{|\alpha|\le m}  \D^\alpha\big(\overline{a_\alpha(x)}v(x)\big)
\end{equation}
gegeben.

\begin{lem}
Der Differentialoperator $\mathcal P$ versehen mit Definitionsbereich
\begin{equation}
  \mathcal D= \{ u\in\rmC^\infty(\Omega) \cap \rmL^2(\Omega) :  \mathcal Pu\in\rmL^2(\Omega) \}
\end{equation}
ist $\rmL^2$-$\rmL^2$-abschließbar.
\end{lem}
\begin{proof}
Sei $u_n\in\mathcal D$ eine Folge mit $u_n\to 0$ in $\rmL^2(\Omega)$ und $\mathcal Pu_n \to w\in\rmL^2(\Omega)$. Dann gilt f\"ur jedes 
$v\in\rmC^\infty_0(\Omega)$
\begin{equation}
 \spro{w}{v}=\lim_{n\to\infty}   \spro{\mathcal Pu_n}{v} = \lim_{n\to\infty} \spro{u_n}{\overline{\mathcal P}v} = 0
\end{equation}
somit $w=0$.
\end{proof}

Die Aussage gilt allgemeiner, es ergibt sich $\rmL^p$-$\rmL^q$-, $\rmC$-$\rmC$ sowie $\rmL^p$-$\rmC$-Abschließbarkeit bei entsprechend gewähltem $\mathcal D$. Insbesondere ist $\mathcal P$ mit Definitionsbereich $\rmC^\infty_0(\Omega)$ abschließbar als Operator auf $\rmL^2(\Omega)$. 

\begin{df} 
Sei $\mathcal P$ ein Differentialausdruck und $\Omega$ ein Gebiet. Dann wird der $\rmL^2$-$\rmL^2$-Abschluß des durch $\mathcal P$ auf $\rmC_0^\infty(\Omega)$ definierten Operators mit $P_0$ und als \eIndex[Differentialoperator]{minimaler Operator} zum Differentialausdruck $\mathcal P$ und Gebiet $\Omega$ bezeichnet. 
Weiterhin heißt $P := (\overline P_0)^*$ \eIndex[Differentialoperator]{maximaler Operator} zu $\mathcal P$ und $\Omega$.
\end{df}

Der so definierte maximale Operator besitzt den Definitionsbereich
\begin{equation}
   \mathcal D_P = \{ u\in\rmL^2(\Omega) : \mathcal Pu \in\rmL^2(\Omega)\},
\end{equation}
wobei die Anwendung von $\mathcal P$ im distributionellen Sinne zu verstehen ist. Wir betrachten ein einfaches Beispiel. Dazu sei $\Omega=(a,b)\subseteq\R$ ein Intervall und $\mathcal P = \D^2 = - \partial^2$ die Zuordnung der zweiten Ableitung. Dann ist besitzt der minimale Operator den Definitionsbereich
\begin{equation}
    \{ u\in\rmH^2(\Omega) :  u(a)=\partial u(a)=u(b)=\partial u(b) = 0\} = \rmH^2_0(\Omega),
\end{equation}
also den Sobolevraum $\rmH^2_0(\Omega)$ der am Rand verschwindenden Funktionen, und der maximale Operator gerade den gesamten Sobolevraum $\rmH^2(\Omega)$ als Definitionsbereich. Im Falle h\"oherer Raumdimensionen wird der Definitionsbereich des maximalen Operators in der Regel echt größer als der Sobolevraum passender Ordnung sein.
Der Unterschied zwischen minimalen und maximalen Operatoren besteht in der Wahl von Randbedingungen. 

\begin{thm}
Die Gleichung $Pu=f$ hat genau dann für jedes $f\in\rmL^2(\Omega)$ eine L\"osung $u\in\mathcal D_{P}$, wenn $\overline P_0$ beschränkt invertierbar ist, also wenn f\"ur den formal adjungierten Differentialausdruck
\begin{equation}
  \forall u\in\rmC^\infty_0(\Omega)\quad:\quad   \|u\| \le C \| \overline{\mathcal P} u \|
\end{equation}
gilt.
\end{thm}
\begin{proof}
Folgt aus Satz~\ref{thm:1:1.2} in Verbindung mit der Definition des maximalen Operators $P$.
\end{proof}

Solche und allgemeinere Ungleichungen stehen im Zusammenhang mit Enthaltentseinsbeziehungen zwischen Definitionsbereichen minimaler Operatoren. Dazu definieren wir zuerst: 
\begin{df}\label{df:1:1.2}
Seien $\mathcal P$ und $\mathcal Q$ Differentialausdrücke und $\Omega$ ein Gebiet. Gilt dann $\mathcal D_{P_0}\subseteq \mathcal D_{Q_0}$, so heiße $\mathcal P$ \eIndex[Differentialoperator]{stärker} als $\mathcal Q$ bzw. $\mathcal Q$ \eIndex[Differentialoperator]{schwächer} als $\mathcal P$. Gilt $\mathcal D_{P_0}=\mathcal D_{Q_0}$, so heißen beide \eIndex[Differentialoperator]{gleich stark}.
\end{df}
Die  Definition hängt im Falle konstanter Koeffizienten nicht von der Wahl des Gebietes ab. Im Falle variabler Koeffizienten (siehe später) sind die Gebiete hinreichend klein zu wählen um eine sinnvolle Definition erhalten.

\begin{lem}
Seien $\mathcal P$ und $\mathcal Q$ Differentialausdr\"ucke auf einem Gebiet $\Omega$. 
\begin{enumerate}
\item[(1)]
$\mathcal Q$ ist genau dann schwächer als $\mathcal P$, wenn
\begin{equation}
    \forall u\in\rmC^\infty_0(\Omega)\quad:\quad \|\mathcal Qu\|^2 \le C \big( \|\mathcal Pu\|^2 + \|u\|^2 \big)
\end{equation}
gilt.
\item[(2)] $Q_0 u$ besitzt für jedes $u\in\mathcal D_{P_0}$ genau dann einen stetigen Repräsentanten, wenn
\begin{equation}
    \forall u\in\rmC^\infty_0(\Omega)\quad:\quad  \sup_{x\in\Omega} |\mathcal Qu(x) |^2 \le C \big( \|\mathcal Pu\|^2 + \|u\|^2 \big)
\end{equation}
gilt.
\end{enumerate}
\end{lem}
\begin{proof}
(1)\quad Die Hinrichtung folgt direkt aus Satz~\ref{thm:1:1.1}. Für die Rückrichtung nehmen wir an, die Ungleichung gilt. Dann existiert zu $u\in\mathcal D_{P_0}$ eine Folge $u_n\in\rmC_0^\infty(\Omega)$ mit $u_n\to u$ und $\mathcal P u_n\to P_0 u$ in $\rmL^2(\Omega)$. Dann ist aber $\mathcal Q(u_n-u_m)$ Cauchy und da $Q_0$ abgeschlossen ist folgt $u\in\mathcal D_{Q_0}$. $\bullet$\qquad
(2)\quad Für die Hinrichtung betrachtet man $\mathcal Q$ als Operator $\mathcal G_{P_0}\to\rmL^\infty(\Omega)$ und wendet Satz~\ref{thm:1:1.1} an. Für die Rückrichtung sei $u\in\mathcal D_{P_0}$ beliebig und $u_n\in\rmC_0^\infty(\Omega)$ eine Folge mit $u_n\to u$ und $\mathcal Pu_n\to P_0u$ in $\rmL^2(\Omega)$.
Dann konvergiert $\mathcal Qu_n$ gleichmäßig und Behauptung folgt.
\end{proof}

\section{Randwertprobleme}
Sei im folgenden $\mathcal P$ und $\Omega$ fixiert. Dann sind  $\mathcal G_P$ und $\mathcal G_{P_0}$ abgeschlossene Teilräume von $\rmL^2\times\rmL^2$. Da
$\mathcal G_{P_0}\subseteq \mathcal G_P$ gilt, kann man den Quotientenraum
\begin{equation}
   \mathcal C = \mathcal G_P / \mathcal G_{P_0}
\end{equation}
betrachten. Dieser wird als \eIndex[Differentialoperator]{Cauchyraum} des Differentialausdrucks $\mathcal P$ auf dem Gebiet $\Omega$ bezeichnet. Für $u\in\mathcal D_P$ sei
\begin{equation}
     \Gamma u := [ (u,Pu) ]_{\mathcal G_{P_0}} \in \mathcal C
\end{equation}
das zugeordnete Cauchydatum. Man kann sich $\Gamma u$ als eine Charakterisierung der Randwerte von $u$ auf $\Omega$ vorstellen, unterscheiden sich zwei Funktionen $u$ und $v$ nur in einer kompakten Teilmenge $\Omega' \Subset \Omega$, so gilt $\Gamma u=\Gamma v$.
Sei nun $B\subseteq\mathcal C$ ein linearer Unterraum. Dann heißt 
\begin{equation}
    Pu = f,\qquad \Gamma u\in B,
\end{equation}
f\"ur gegebenes $f\in\rmL^2(\Omega)$ das zugeordnete \eIndex{Randwertproblem} mit homogener \eIndex[Randwertproblem]{Randbedingung} $\Gamma u\in B$. Die Randbedingung 
$B$ bestimmt damit eine Einschr\"ankung $\widehat P$ des Operators $P$ auf den Definitionsbereich
\begin{equation}
    \mathcal D_{\widehat P} = \{ u \in \mathcal D_P : \Gamma u\in B \}.
\end{equation}
Das Randwertproblem heißt \eIndex[Randwertproblem]{korrekt gestellt}, falls $\widehat P$ stetig invertierbar ist.

\begin{thm}
Zu einem Differentialausdruck $\mathcal P$ existieren auf einem Gebiet $\Omega$ genau dann korrekt gestellte Randwertprobleme, wenn $P_0$ und $\overline P_0$
beschr\"ankt invertierbar sind.
\end{thm} 
\begin{proof} 
Angenommen, es existiert ein korrekt gestelltes homogenes Randwertproblem $\widehat P$. Da $P_0\subseteq \widehat P$ gilt, muß dann $P_0^{-1}$ beschränkt sein. Weiterhin ist $\widehat P^{-1}$ rechtsinvers zu $P$, damit muß $\overline P_0^{-1}$ nach Korollar~\ref{cor:1:1.3} beschränkt sein.

Seien nun $P_0$ und $\overline P_0$ beschränkt invertierbar. Da $P_0^{-1}$ beschr\"ankt ist, ist $\mathcal R_{P_0}\subseteq \rmL^2(\Omega)$ abgeschlossen.  Bezeichne nun $\pi$ den Orthogonalprojektor auf $\mathcal R_{P_0}$. Ist nun $S$ eine Rechtsinverse zu $P$ (die nach  Korollar~\ref{cor:1:1.3} existiert), so gilt für
den durch  
\begin{equation}
  Tf = P_0^{-1} \pi f + S(I-\pi) f,\qquad f\in\rmL^2(\Omega) 
\end{equation}
definierten Operator $T$ nach Konstruktion $PT f = \pi f + (I-\pi) f = f$ und $T$ besitzt eine Inverse $\widehat P \subseteq P$. Weiter ist $T\supset P_0^{-1}$ und $\widehat P\supset P_0$ , so daß $P_0\subseteq \widehat P\subseteq P$. Da $\widehat P^{-1}$ beschränkt und auf ganz $\rmL^2(\Omega)$ definiert ist, ist der Satz bewiesen.
\end{proof}

\section{Differentialoperatoren mit konstanten Koeffizienten}
Im folgenden sollen vorerst nur Operatoren mit konstanten Koeffizienten betrachtet werden. Diese gehören zu Differentialausdrücken der Form
\begin{equation}
     P(\D) = \sum_{|\alpha|\le m} a_\alpha \D^\alpha
\end{equation}
mit $a_\alpha\in\C$. Für diese gilt 
\begin{equation}
     P(\D) \e^{\i x\cdot\zeta}  =  P(\zeta) \e^{\i x\cdot\zeta}
\end{equation}
für alle $\zeta\in\C^n$ und mit $x\cdot\zeta = \sum_{j=1}^n x_j\zeta_j$. Das Polynom $ P(\zeta)$ heißt das \eIndex[Differentialoperator]{Symbol} des Differentialausdrucks $\mathcal P(\D)$. Ist nun $\Omega\subseteq\R^n$ beschränkt und $u\in\rmC_0^\infty(\Omega)$, so ist die Fourier--Laplace-transformierte von $u$
\begin{equation}
    \widehat u (\zeta) = (2\pi)^{-n/2} \int_\Omega \e^{-\i x\cdot\zeta } u(x) \d x,\qquad \zeta\in\C^n
\end{equation}
eine ganze Funktion auf $\C^n$. Die Anwendung des Operators $ P(\D)$ entspricht im Fourierbild der Multiplikation mit $ P(\zeta)$. 

Eine erste Anwendung ist folgender Satz. Ein Beweis ergibt sich später nochmals in allgemeinerer Form.

\begin{thm}\label{thm:1:1.8}
Der Operator $P_0$ ist beschränkt invertierbar, es existiert also eine Konstante $C$ mit
\begin{equation}
   \forall u\in\rmC_0^\infty(\Omega)\quad:\quad \|u\|\le C\| P(\D) u\|.
\end{equation}
\end{thm}

Der Beweis basiert auf folgendem Lemma der Funktionentheorie. 
\begin{lem}
Sei $g\in\mathfrak A(\overline{\mathbb D})$ analytisch in einer Umgebung der Kreisscheibe $|z|\le 1$  und $r$ ein Polynom mit höchstem Koeffizienten $A$. Dann gilt
\begin{equation}
   |Ag(0)|^2 \le (2\pi)^{-1} \int_0^{2\pi} |g(\e^{\i\theta}) r(\e^{\i\theta})|^2 \d\theta.
\end{equation}
\end{lem}
\begin{proof}
Seien $z_j$ die Nullstellen von $r$ innerhalb der Kreisscheibe $|z|<1$ und
\begin{equation}
    r(z) = q(z) \prod_{j} \frac{z_j-z}{\overline{z_j}z-1}.
\end{equation} 
Dann gilt auf der Kreislinie $|r(z)|=|q(z)|$ und $q(z)$ ist analytisch in $\mathbb D$. Also folgt
\begin{equation}
  (2\pi)^{-1} \int  |g(\e^{\i\theta}) r(\e^{\i\theta})|^2 \d\theta = (2\pi)^{-1} \int  |g(\e^{\i\theta}) q(\e^{\i\theta})|^2 \d\theta \ge |g(0)q(0)|^2.
\end{equation}
Nun ist aber $q(0)/A$ gerade das Produkt der Nullstellen von $r(z)$ außerhalb des Einheitskreises und damit $|q(0)|\ge |A|$ und das Lemma folgt.
\end{proof}

\begin{proof}[Beweis zu Satz~\ref{thm:1:1.8}]
Bezeichne $p(\zeta)$ den homogenen Hauptteil von $P(\zeta)$. Sei weiter $\xi_0\in\R^n$ mit $p(\xi_0)\ne0$. Wendet man nun obiges Lemma auf die Funktion
$\widehat u(\zeta+ t\xi_0)$ und das Polynom $P(\zeta+t\xi_0)$ (verstanden als Funktionen von $t$) an, so erhält man
\begin{equation}
   |\widehat u(\zeta) p(\xi_0)|^2 \le (2\pi)^{-1} \int |\widehat u(\zeta+\e^{\i\theta} \xi_0) P(\zeta+\e^{\i\theta}\xi_0)|^2 \d\theta.
\end{equation}
Speziell mit $\zeta=\xi\in\R^n$ und Integration bezüglich $\xi$ ergibt sich
\begin{align}
   |p(\xi_0)|^2 \int |\widehat u(\xi)|^2 \d\xi &\le (2\pi)^{-1} \iint |\widehat u(\xi+\e^{\i\theta}\xi_0) P(\xi+\e^{\i\theta}\xi_0)|^2 \d\xi\d\theta \notag\\
   &= (2\pi)^{-1} \iint |\widehat u(\xi+\i\xi_0\sin\theta) P(\xi+\i\xi_0\sin\theta)|^2 \d\xi\d\theta
\end{align} 
und unter Ausnutzung der Parseval-Identität
\begin{equation}
   |p(\xi_0)|^2 \int |u(x)|^2\d x \le (2\pi)^{-1} \iint |P(D) u(x)|^2 \e^{2x\cdot\xi_0 \sin\theta} \d x\d\theta
\end{equation}
und mit $C=\sup_{x\in\Omega} \e^{|x\cdot\xi_0|} /|p(\xi_0)|$ folgt die Behauptung.
\end{proof}

\begin{cor}\label{cor:1:1.10}
Zu den Differentialoperatoren mit konstanten Koeffizienten existieren
für beliebige Gebiete korrekt gestellte Randwertprobleme.
\end{cor}
 % Einleitung, Teil 1
% !TEX root = main.tex
\chapter{Minimale Operatoren}

Als erste Etappe in diesem Abschnitt führen wir den Vergleich der Stärke (Definition \ref{df:1:1.2})
zweier Differentialperatoren mit konstanten Koeffizienten,
auf einen Vergleich der Symbole zurück.

\section{Die Stärke von Differentialausdrücken}

Es seien in diesem Abschnitt stets $\mcP=P(D)$, $\mcQ=Q(D)$ und $\Om$ ein beschränktes Gebiet.
Ziel ist die Charaktierisierung der Aussage ``$\mathcal Q$ ist schächer als $\mcP$''
durch eine Ungleichung der Form
\begin{equation}\label{eq:2:1}
\sup_{\x\in\R^n}\frac{\til{Q}(\x)}{\til{P}(\x)}<\infi,
\end{equation}
wobei $\til{P}$ und $\til{Q}$ geeignete Regularisierungen von $P$ und $Q$ sind.
Beachte, dass die reellen Nullstellen von $P$ Probleme machen würden.

Im eindimensionalen Fall ist bekannt, dass $Q(D)$ genau dann schwächer wie $P(D)$ ist,
wenn $P(\z)$ höheren oder gleichen Grad wie $Q(\z)$ hat.
%Zum Beweis genügt die mehrfache Anwendung des Theorems \ref{thm:1:1.8}
%auf den Operator $D$ und der Dreiecksungleichung.
Als Regularisierungen könnte man $\til{P}(\z)=\sqrt{1+\abs{P(\z)}^2}$
und $\til{Q}(\z)=\sqrt{1+\abs{Q(\z)}^2}$ verwenden.
Daß dies schon im zweidimensionalen nicht mehr Funktioniert
hängt damit zusammen, dass $\abs{P^{(\al)}(\z)}/\sqrt{1+\abs{P(\z)}^2}$
nun im allgemeinen nicht mehr beschränkt ist.
Man betrachte zum Beispiel $P(\z)=\z_1\z_2$ und $P^{(1)}(\z)=\z_2$.

Es zeigt sich, dass die Regularisierungen
\begin{equation}\label{eq:2:2}
\til{P}(\z):=\sqrt{\sum_\al\abs{P^{(\al)}(\z)}^2},
\quad\text{f.a.}~\z\in\C^n,
\end{equation}
eine natürliche Wahl sind.
Also der Absolutbetrag des Vektors $(P^{(\al)}(\z))_\al\in\C^M$.
Beachte, dass für $\x\in\R^n$ der Ausdruck
$\til{P}(\x)^2$ ein reelles Polynom ohne Nullstellen ist.

\begin{thm}\label{thm:2:2.1}
Der Differentialoperator $\mcQ$ ist genau dann schwächer wie $\mcP$,
wenn
\begin{equation}\label{eq:thm:2:2.1}
\sup_{\x\in\R^n}\frac{\til{Q}(\x)}{\til{P}(\x)}<\infi
\end{equation}
erfüllt ist.
\end{thm}

Wir zeigen zunächst, wie in beiden Richtungen abgeschätzt wird.
Die Hinrichtung ist straight forward,
für die Rückrichtung benötigen wir ein Lemma,
dass wir später beweisen.

\begin{proof}
{\em Hinrichtung}:
Sei $\mcQ$ schwächer wie $\mcP$.
Dann gilt für ein $C\in(0,\infi)$ die Ungleichung
\begin{equation}\label{eq:2:4}
\norm{\mcQ u}\leq C(\norm{\mcP u}^2+\norm{u}^2),\quad\text{f.a.}~u\in C^\infi_0(\Om).
\end{equation}
Für jedes $\x\in\R^n$ und ein festes $\psi\in C^\infi_0(\Om)$, $\psi\neq0$,
definieren wir $\psi_\x(x):=\psi(x)\e^{\ska{x}{\x}}$, f.a.~$x\in\R^n$.
Dies bringt durch die Formel von Leibniz die Ableitungen von $P(\x)$ ins Spiel
\begin{equation}\label{eq:2.5}
P(D)\psi_\x(x)=\e^{\ska{x}{\x}}\sum_\al P^{(\al)}(\x)\frac{D_\al\psi(x)}{\al!}.
\end{equation}
Schreiben wir kurz
\begin{equation}\label{eq:2.6}
\Psi_{\al\be}:=\frac1{\al!\be!}\spro{D_\al\psi}{D_\be\psi},
\end{equation}
so liest sich die Ungleichung \eqref{eq:2:4} mit $u=\psi_\x$ als
\begin{equation}\label{eq:2.7}
\sum_{\al,\be}Q^{(\al)}(\x)\bar{Q}^{(\be)}(\x)\Psi_{\al\be}
\leq C\left(\sum_{\al,\be}P^{(\al)}(\x)\bar{P}^{(\be)}(\x)\Psi_{\al\be}+\Psi_{00}\right).
\end{equation}
Da $\psi_\x\in C^\infi_0$, $\psi_\x\neq0$ gilt, sind $\norm{\mcP\psi_\x}$ und $\norm{\mcQ\psi_\x}$ stets positiv.
Somit ist $\left(\Psi_{\al\be}\right)_{\al,\be}$ auch die gram'sche Matrix
eines Skalarproduktes auf einem $\C^M$.
Es gibt also $\Psi_-,\Psi_+\in(0,\infi)$, so dass
\begin{equation}\label{eq:2.8}
\abs{\z}^2\Psi_-\leq\sum_{\al,\be}\z_\al\z_\be\Psi_{\al\be}\leq\abs{\z}^2\Psi_+,
\quad\text{f.a.}~\z\in\C^M.
\end{equation}
Die Ungleichungen \eqref{eq:2.7} und \eqref{eq:2.8} kombinieren sich zu
\begin{equation}\label{eq:2.9}
\til{Q}(\x)^2\Psi_-C\leq\Psi_+\til{P}(\x)^2,
\quad\text{f.a.}~\x\in\R^n,
\end{equation}
und dies ist gerade die Ungleichung \eqref{eq:thm:2:2.1}.

{\em Rückrichtung}:
Sei die Ungleichung \eqref{eq:thm:2:2.1} erfüllt, d.h.
\begin{equation}\label{eq:2.10}
\til{Q}(\x)\leq C\til{P}(\x),
\quad\text{f.a.}~\x\in\R^M,
\end{equation}
mit $C\in(0,\infi)$.
Wir schätzen wie folgt ab
\begin{equation}\label{eq:2.11}
\norm{Q(D)u}^2=\norm{Q\cdot\hat{u}}^2\leq C\norm{\til{P}\cdot\hat{u}}^2
\leq C\sum_{\al}\norm{P^{(\al)}\cdot\hat{u}}^2=C\sum_\al\norm{P^{(\al)}(D)u}^2.
\end{equation}
Um den Beweis abzuschließen fehlte also nur noch eine Ungleichung
\begin{equation}
\norm{P^{(\al)}(D)u}\leq C_{\al,\mcP}\norm{P(D)u},
\quad\text{f.a.}~u\in C^\infi_0(\Om),
\end{equation}
mit $C>0$.
Diese werden wir mit der Methode der Energieintegrale herleiten (Korollar \ref{cor:2:2.5}).
\end{proof}

Der Beweis zeigt,
dass in Formel \eqref{eq:thm:2:2.1} $\til{Q}$ durch $Q$ ersetzt werden kann.

\section{Energieintegrale}

Da wir eine Methode brauchen um Skalarprodukte
mit verschiedenen Differentialoperatoren abzuschätzen,
macht es Sinn, quadratische Differentialformen $F$, definiert durch
\begin{equation}
F(D,\bar{D})u\bar{u}:=\sum_{\al,\be}a_{\al\be}(D^\al u)(\cc{D^\be u}),
\quad\text{f.a.}~u\in C^\infi_0(\Om),
\end{equation}
mit $a_{\al\be}\in\C$, genauer zu betrachten.
Uns interessieren die Terme
\begin{equation}\label{eq:2:qf}
\int F(D,\bar{D})u(x)\bar{u}(x)\d x=\int F(\x,\x)\abs{\hat{u}(\x)}^2\d \x,
\quad\text{f.a.}~u\in C^\infi_0(\Om),
\end{equation}
die quadratische Formen \eqref{eq:2:qf} definieren.
Die Formen sind, wie die Gleichung zeigt,
durch das Symbol $F(\z,\cc{\z})$, $\z\in\C^n$ bestimmt.
Umgekehrt sind die reellen Werte des Symbols
durch die Form \eqref{eq:2:qf} bestimmt,
wie folgendes Lemma zeigt.
Im allgemeinen können verschiedene Symbole
die gleiche Form \eqref{eq:2:qf} definieren.

\begin{lem}\label{lem:2:qfsym}
Sei $\Om$ ein beliebiges Gebiet.
Die Gleichung
\begin{equation}\label{eq:lem:2:qfsym}
\int F(D,\bar{D})u\bar{u}\d x=0,
\quad\text{f.a.}~u\in C^\infi_0(\Om),
\end{equation}
gilt genau dann, wenn $F(\x,\x)=0$, f.a.~$\x\in\R^n$.
\end{lem}
\begin{proof}
Es gelte \eqref{eq:lem:2:qfsym}.
Sei $u_\y(x):=u(x)\e^{\ska{x}{\y}}$ für ein festes $u\in C^\infi_0(\Om)$, $u\neq0$.
Dann ist
\begin{equation}
\int F(D,\bar{D})u_\y\bar{u_\y}\d x=\int F(\x,\x)\abs{\hat{u}(\x-\y)}^2\d \y=\int F(\x+\y,\x+\y)\abs{\hat{u}(\x)}^2\d \x=:q(\y),
\end{equation}
für alle $\y\in\R^n$.
Auf der rechten Seite steht wieder ein Polynom $q(\y)$ in $\y\in\R^n$
und nach Voraussetzung müssen dessen Koeffizienten verschwinden.
Beachten wir, dass
\begin{equation}
(\x+\y)^\al=\sum_{\be+\ga=\al}\begin{pmatrix}\al\\\be,\ga\end{pmatrix}\x^\be\y^\ga=\y^\al+\dots
\end{equation}
so sehen wir, dass die führenden Terme von $q(\y)$
bis auf den Faktor $\norm{u}^2\neq0$ den führenden Termen von $F(\x,\x)$ entsprechen.
Wäre also $F(\x,\x)\neq0$, so hätten wir einen Widerspruch.

Die Rückrichtung sahen wir bereits, siehe Gleichung \eqref{eq:2:qf}.
\end{proof}

Unser Ziel ist es, Ableitungen von Ableitungspolynomen zu kontrollieren.
Dazu betrachten wir Ableitungen quadratischer Differentialformen.
Wir beginnen mit dem Ausdruck
\begin{equation}
\frac{\d}{\partial x_k}\left[F(D,\bar{D})u\bar{u}\right]=\i\left[(D_k-\bar{D}_k)F(D,\bar{D})\right]u\bar{u}.
\end{equation}
Für einen Vektor $\ul{G}=(G_k)_{k=1,\dots,n}$, bestehend aus quadratischen Differentialformen $G_k$,
erhält man für die Divergenz die Formel
\begin{subequations}
\begin{equation}\label{eq:2.15a}
\div(\ul{G}(D,\bar{D})u\bar{u})=\sum_{k=1}^n\frac{\d}{\partial x_k}\left[G_k(D,\bar{D})u\bar{u}\right]=F(D,\bar{D})u\bar{u},
\end{equation}
wobei
\begin{equation}\label{eq:2.15b}
F(\z,\bar{\z})=\i\sum_{k=1}^n(\z_k-\bar{\z}_k)G_k(\z,\bar{\z})=-2\sum_{k=1}^n\y_kG_k(\z,\bar{\z}),
\end{equation}
\end{subequations}
wenn wir $\z=\x+\i\y$ schreiben.
\begin{lem}\label{lem:2:2.2}
Ein Polynom $F(\z,\bar{\z})$ in $\z=\x+\i\y$ und $\bar{\z}=\x-\i\y$
kann genau dann in der Form \eqref{eq:2.15b} dargestellt werden,
wenn $F(\x,\x)=0$ für alle $\x=0$.

In diesem Fall gilt
\begin{equation}\label{eq:2.16}
G_k(\x,\x)=-\tfrac12\left.\frac{\partial F(\x+\i\y,\x-\i\y)}{\partial \y_k}\right\rvert_{\y=0}.
\end{equation}
\end{lem}
\begin{proof}
Die Hinrichtung ist offensichtlich.
Die Rückrichtung folgt aus der (reellen) Taylorentwicklung von $F(\x+\i\y,\x-\i\y)$ in $(\x,\y)\in\R^{2n}$.
Der Satz von Taylor angewendet auf variables $\y\in\R^n$ und festes $\x\in\R^n$ liefert schließlich \eqref{eq:2.16}.
\end{proof}

Die Polynome $G_k(\z,\bar{\z})$, $\z\in\C^n$, sind als ganzes sind nicht eindeutig festgelegt.
Doch aufgrund von Lemma \ref{lem:2:qfsym} stört das nicht.

Mit Hilfe einer speziellen quadratischen Differentialformen und obiger Formel
gewinnen wir folgende Ungleichung,
mit der wir die Ableitungen der Differentialoperatoren kontrollieren können.

\begin{lem}
Sei $B_k:=\sup_{x_k,y_k\in\Om}\abs{x_k-y_k}<\infi$.
Für Ableitungspolynome $P(D)$, $Q(D)$ gilt die Ungleichung
\begin{equation}\label{eq:lem:2:2.4}
\abs{\spro{P^{(k)}(D)u}{\bar{Q}(D)u}}\leq\norm{P(D)u}\left(\norm{\bar{Q}^{(k)}(D)u}+B_k\norm{\bar{Q}(D)u}\right),
\quad\text{f.a.}~u\in C^\infi_0(\Om).
\end{equation}
\end{lem}

\begin{proof}
Wir betrachten die quadratische Differentialform mit dem Symbol
\begin{equation}
F(\z,\bar{\z}):=P(\z)Q(\bar{\z})-Q(\z)P(\bar{\z}).
\end{equation}
Diese erfüllt offensichtlich $F(\x,\x)=0$, f.a.~$\x\in\R^n$.
Nach Lemma \ref{lem:2:2.2} und der Formel \eqref{eq:2.16}
erhalten wir, nach dem wir mit $-\i x_k$ multipliziert haben
\begin{equation}\label{eq:2:2.22}
-\i x_kF(D,\bar{D})u\bar{u}=-\i x_k\sum_{k=1}^n\frac{\d}{\partial x_k}\left[G_k(D,\bar{D})u\bar{u}\right],
\end{equation}
mit
\begin{equation}
G_k(\x,\x)=-\i\left(P^{(k)}(\x)Q(\x)-Q^{(k)}(\x)P(\x)\right),
\quad\text{f.a.}~\x\in\R^n.
\end{equation}

Integrieren wir die Gleichung \eqref{eq:2:2.22},
so liefert eine partielle Integration auf rechten Seite den Term
\begin{align}
\i\int G_k(D,\bar{D})u\bar{u}\d x&=\int\left(P^{(k)}(D)Q(\bar{D})-P(D)Q^{(k)}(D)\right)u\bar{u}\d x\\
&=\spro{P^{(k)}(D)u}{\bar{Q}(D)u}-\spro{P(D)u}{\bar{Q}^{(k)}(D)u}.
\end{align}
Damit erhalten wir insgesamt
\begin{equation}
\spro{P^{(k)}(D)u}{\bar{Q}(D)u}
=\spro{P(D)u}{\bar{Q}^{(k)}(D)u}-i\int x_k\left(P(D)u\cc{\bar{Q}(D)u}-Q(D)\cc{\bar{P}(D)u}\right)\d x.
\end{equation}
Ohne Beschränkung der Allgemeinheit können wir das Gebiet $\Om$ so legen,
dass $\abs{x_k}\leq B_k/2$ für alle $x_k\in\Om$.
Man beachte dass Ableitungspolynome mit Translationsoperatoren vertauschen.
Schließlich erhalten wir mit Cauchy-Schwarz
und der Dreiecksungleichung die Ungleichung \eqref{eq:lem:2:2.4}.
\end{proof}

\begin{cor}\label{cor:2:2.5}
Ist $P(\x)$ vom Grad $m$ in $\x_k$, so gilt
\begin{equation}\label{eq:cor:2:2.5}
\norm{P^{(k)}(D)u}\leq mB_k\norm{P(D)u},
\quad\text{f.a.}~u\in C^\infi_0(\Om).
\end{equation}
\end{cor}
\begin{proof}
Setzen wir $\bar{Q}(\x)=P^{(k)}(\x)$, so erhalten wir aus \eqref{eq:lem:2:2.4} die Ungleichung
\begin{equation}\label{eq:2.28}
\norm{P^{(k)}(D)u}^2\leq\norm{P(D)u}\left(\norm{P^{(kk)}(D)u}+B_k\norm{P^{(k)}(D)u}\right),
\quad\text{f.a.}~u\in C^\infi_0(\Om).
\end{equation}
Wir gehen nun Induktiv in $m$ vor.
Für $m=1$ ist $P^{(kk)}=0$, $P^{(k)}\neq0$ und somit entspricht die Gleichung \eqref{eq:cor:2:2.5}
gerade der Gleichung \eqref{eq:lem:2:2.4} nach kürzen eines Faktors.
Angenommen das Korollar ist für $m-1$ erfüllt,
dann erhalten durch Kombination von \eqref{eq:2.28} und \eqref{eq:cor:2:2.5}
\begin{equation}
\norm{P^{(k)}(D)u}^2\leq\norm{P(D)u}\left((m-1)B_k\norm{P^{(k)}(D)u}+B_k\norm{P^{(k)}(D)u}\right),
\end{equation}
also ist auch der Induktionsschritt gezeigt.
\end{proof}

\begin{cor}\label{cor:2:2.6}
Für festes beschränktes Gebiet $\Om$ und beliebiges $\al\in\N^n_0$ gilt
\begin{equation}
\norm{P^{(\al)}(D)u}\leq C_\al\norm{P(D)u},
\quad\text{f.a.}~u\in C^\infi_0(\Om),
\end{equation}
mit $C_\al=m(m-1)\cdots(m-\abs{\al})B^\al$, wobei $B^\al=\prod_{l=1}^n B^{\al_l}_l$.
Die Konstante $C_\al$ hängt also nur vom Grad $m$ von $P(\x)$,
der Ableitungsordnung $\abs{\al}$ und der Ausdehnung des Gebiets $\Om$ ab.
\end{cor}

\begin{proof}
Mehrfache Anwendung von Korollar \ref{cor:2:2.5}.
\end{proof}

\section{Beispiele und spezielle Symbole}

Wir betrachten drei instruktive klassische Differentialoperatoren zweiter Ordnung.

\begin{exa}\label{exa:lap}
Wir betrachten das zum Laplace-Operator $\Lap$
korrespondierende Symbol $P(\x)=\x_1^2+\dots+\x_n^2$.
Das regularisierte Symbol ist gegeben durch
\begin{equation}
\abs{\til{P}(\x)}^2=(\x^2_1+\dots+\x^2_n)^2+4(\x_1^2+\dots+\x_n^2)+4n.
\end{equation}
Beachten wir, dass stets $(a+b)^2\leq2(a^2+b^2)$ für $a,b\in\R$,
so sehen wir, dass
\begin{equation}
\abs{\til{P}(\x)}^2\asymp(\x^2_1+\dots+\x^2_n)^2+1.
\end{equation}
Also ist der Laplaceoperator stärker wie alle Differentialoperatoren
mit Symbolen von gleichem oder kleinerem Grad.
Diese sind alle von der Form
\begin{equation}
Q(\x)=\x^\tr A\x+\ska{b}{\x}+c,
\end{equation}
mit einer Matrix $A\in\C^{n\ti n}$, $b\in\C^n$ und $c\in\C$,
wobei wir $A^\tr=A$ annehmen können.
Wir untersuchen, welche davon gleich stark wie der Laplaceoperator sind.
Es ist
\begin{align}
\abs{\til{Q}(\x)}^2&=\abs{\x^\tr A\x+\ska{b}{\x}+c}^2+\sum_{k=1}^n\abs{2e^\tr_kA\x+b_k}^2
+4\sum_{1\leq k\leq l\leq n}\abs{a_{kk}}^2,\\
&\asymp\abs{\x^\tr A\x}^2+\abs{\ska{b}{\x}}^2+\sum_{k=1}^n\abs{e^\tr_kA\x}^2+1.
\end{align}
Wegen
\begin{equation}
\frac{\abs{\ska{b}{\x}}^2+\sum_{k=1}^n\abs{e^\tr_kA\x}^2+1}{(\x^2_1+\dots+\x^2_n)^2+1}\to0,
\quad\text{falls}~\x\to0
\end{equation}
hängt alles vom Term $\abs{\x^\tr A\x}^2$ ab.
\end{exa}

\begin{exa}\label{exa:schroe}
Der Schrödingergleichung für ein freies Teilchen
\begin{equation}
\i\frac{\d}{\partial t}\psi(x,t)=-\Lap\psi(x,t),\end{equation}
lässt sich das Symbol $P(\x)=\x^2_1+\dots+\x^2_{n-1}-\x_n$ zuordnen.

Wir bestimmen alle Operatoren gleicher stärke.
Das Symbol $Q(\x)$ ist genau dann schwächer wie $P(\x)$,
wenn
\begin{equation}\label{eq:2:subschroe}
\abs{Q(\x)}^2\apprle(\x^2_1+\dots+\x^2_{n-1}-\x_n)^2+\x^2_1+\dots+\x^2_{n-1}+1.
\end{equation}
Also hat $Q(\x)$ maximalen Grad 2 in $\x_1,\dots,\x_{n-1}$,
Grad 1 in $\x_n$ und somit von der Form
\begin{equation}
Q(\x)=a_0+\sum^n_{k=1}a_k\x_k+\sum^n_{k,l=1}a_{kl}\x_k\x_l,
\end{equation}
mit $a_{nn}=0$.
Wir betrachten die Gleichung \eqref{eq:2:subschroe}
für spezielle Vektoren:
\begin{equation}\label{eq:2:subschroespec}
\abs{Q(\x)}^2\apprle(\x^2_1+\dots+\x^2_{n-1}+1),
\quad\text{für alle}~\x\in\R^n~\text{mit}~\x_n=\x^2_1+\dots+\x^2_{n-1}.
\end{equation}
Da wir $\x_1,\dots,\x_{n-1}$ in \eqref{eq:2:subschroespec} frei wählen können folgt, dass
\begin{equation}
Q(\x)=a_0+\sum_{k=1}^{n-1}a_k\x_k+a_n(\x_n-\x^2_1-\dots-\x^2_{n-1}),
\end{equation}
mit beliebigen $a_0,a_1,\dots,a_n\in\C$.
Es ist $Q(\x)$ genau dann gleich Stark wie $P(\x)$,
wenn $a_n\neq0$.
\end{exa}

\begin{exa}\label{exa:heat}
Der Wärmeleitungsgleichung
\begin{equation}
\Lap T(x,t)=\frac{\d}{\partial t}T(x,t),
\end{equation}
entspricht das Symbol $P(\x)=\x^2_1+\dots+\x^2_{n-1}+\i\x_n$.

Ein Symbol $Q(\x)$ ist genau dann stärker wie $P(\x)$, wenn
\begin{equation}
\abs{Q(\x)}^2\apprle(\x^2_1+\dots+\x^2_{n-1})^2+\x^2_1+\dots+\x^2_{n-1}+\x^2_n+1.
\end{equation}
Diese Ungleichung ist genau dann erfüllt, wenn
\begin{equation}
Q(\x)=a_0+\sum_{k=1}^na_k\x_k+\sum_{k,l=1}^{n-1}a_{kl}\x_k\x_l,
\end{equation}
mit beliebigen $a_{kl}\in\C$, $k,l=1,\dots,n-1$.
Das Symbol $Q(\x)$ ist genau dann gleich stark wie $P(\x)$,
wenn $a_n\neq0$ und die Matrix $(a_{kl})_{k,l=1,\dots,n-1}$ positiv ist
(diese wird ohne Beschränkung der Allgemeinheit symmetrisch gewählt).
\end{exa}

\begin{exa}\label{exa:hyper}
Als letztes Beispiel betrachten wir die Gleichung
\begin{equation}
\Lap_1u=\Lap_2u,
\end{equation}
mit $\Lap_1:=\d_1^2+\dots+\d_m^2$, $\Lap_2:=\d^2_{m+1}+\dots+\d^2_m$.
Das korrespondierende Symbol ist $P(\x)=\x^2_1+\dots+\x^2_m-\x^2_{m+1}-\dots-\x^2_n$.

Ein Symbol $Q(\x)$ ist genau dann gleich stark wie $P(\x)$,
wenn
\begin{equation}
\abs{Q(\x)}^2\apprle(\x^2_1+\dots+\x^2_m-\x^2_{m+1}+\dots+\x^2_n)^2+\x^2_1+\dots+\x^2_n+1.
\end{equation}
Setzen wir wie in Beispiel \ref{exa:schroe} $\x^2_1+\dots+\x^2_m=\x^2_{m+1}+\dots+\x^2_n$,
so erhalten wir für diese $\x$:
\begin{equation}
\abs{Q(\x)}^2\apprle\x^2_1+\dots+\x^2_n+1.
\end{equation}
Es folgt, dass $Q(\x)$ von der Form
\begin{equation}
Q(\x)=a_0+\sum_{k=1}^na_k\x^k+b(\x_1^2+\dots+\x^2_m-\x^2_{m+1}+\dots+\x^2_n),
\end{equation}
mit beliebigen $a_0,a_1,\dots,a_n,b\in\C$ sein muss
und $Q(\x)$ ist genau dann gleich stark wie $P(\x)$,
wenn $b\neq0$.
\end{exa}

\section{Weitere Vergleichsresultate}

Sehr analog zu Satz \ref{thm:2:2.1} lassen sich die regularisierten
Symbole $\til{P}$, $\til{Q}$ heranziehen
um die Kompaktheit der Abbildung
\begin{equation}\label{eq:2:map}
\mcR_{P_0}\to\mcR_{Q_0},\quad P(D)u\mto Q(D)u,
\end{equation}
zu charakterisieren (falls sie existiert).
Existiert die Abbildung \eqref{eq:2:map} und ist kompakt,
so nennen wir $Q(D)$ relativ kompakt zu $P(D)$.
Oben hatten wir charakterisiert,
wann die Abbildung \eqref{eq:2:map} existiert und stetig ist.

\begin{thm}
Der Operator $Q(D)$ ist genau dann relativ kompakt zu $P(D)$,
wenn
\begin{equation}\label{eq:2:tozero}
\frac{\til{Q}(\x)}{\til{P}(\x)}\to0,\quad\text{für}\quad\x\to\infi.
\end{equation}
\end{thm}
\begin{proof}
Sei die Gleichung \eqref{eq:2:tozero} erfüllt.
Wir wählen eine beliebige Folge $u_n\in C^\infi_0(\Om)$,
so dass 
\begin{equation}\label{eq:2:PDb}
\norm{P(D)u_n}\leq1.
\end{equation}
Wir zeigen, dass es eine Teilfolge $u_{n^\pr}$ gibt,
so dass $Q(D)u_{n^\pr}$ konvergiert.
Zunächst stellen wir fest, dass der Ausdruck $\til{Q}(\x)/\til{P}(\x)$
stets lokal-beschränkt ist.
Somit impliziert \eqref{eq:2:tozero} bereits \eqref{eq:thm:2:2.1}
und nach Satz \ref{thm:2:2.1} ist damit
\begin{equation}
\norm{Q(D)u_n}\leq C,
\quad\text{f.a.}~n\in\N,
\end{equation}
für ein $C\in(0,\infi)$.
Nach Hölder und der Stetigkeit der Fouriertransformation
von $L^1$ nach $L^\infi$ impliziert dies
\begin{equation}
\sqrt{2\pi}\norm{D^\al(Q(\x)u_n)}_\infi\leq\norm{x^\al Q(D)u_n}_1\leq\norm{Q(D)u_n}\norm{x^\al1_\Om}
\leq\norm{Q(D)u_n}\sqrt{\abs{\Om}}\tfrac{B^\al}{2^{\abs{\al}}}.
\end{equation}
In Kombination mit dem Riemann-Lebesgue-Lemma folgt,
dass alle Ableitungen von $Q(\x)\hat{u}_n(\x)$
gleichmäßig beschränkt sind und im unendlichen abfallend.
Insbesondere lässt sich eine lokal-gleichmäßig konvergente Teilfolge $Q(\x)\hat{u}_{n^\pr}(\x)$ auswählen.

Wir zeigen, dass dies unser Kandidat ist.
Nach Voraussetzung lässt sich zu jedem $\ep>0$ eine kompakte Menge $K$ wählen,
so dass $\abs{Q(\x)}/\til{P}(\x)<\ep$ für $\x$ im Komplement $K^{\mathrm c}$ von $K$.
Nun ist
\begin{align}
\int_{K^{\mathrm c}}\abs{Q(\x)}^2\abs{\hat{u}_{n^\pr}(\x)-\hat{u}_{m^\pr}(\x)}^2\d \x,
&\leq\ep^2\int\til{P}(\x)^2\abs{\hat{u}_{n^\pr}(\x)-\hat{u}_{m^\pr}(\x)}^2\d \x\\
&=\ep^2\sum_{\al\in\N^n_0}\norm{P^{(\al)}(D)(u_{n^\pr}-u_{m^\pr})}^2,\label{eq:2:Kcsum}
\end{align}
und die Summe in \eqref{eq:2:Kcsum} ist nach Korollar \ref{cor:2:2.6}
und \eqref{eq:2:PDb} beschränkt.
Weiter gilt
\begin{equation}
\int_K\abs{Q(\x)}^2\abs{\hat{u}_{n^\pr}(\x)-\hat{u}_{m^\pr}(\x)}^2\d \x\to0,
\quad\text{falls}~n^\pr~\text{und}~m^\pr\to\infi,
\end{equation}
aufgrund der lokal-gleichmäßigen Konvergenz.
Da $\ep>0$ beliebig gewählt werden konnte,
impliziert dies die Konvergenz von $Q(D)u_{n^\pr}$ in $L^2$.

Nehmen wir nun an $Q(D)$ ist kompakt relativ zu $P(D)$.
Dazu weisen wir nach, dass aus $\x_n\to\infi$ die Konvergenz $\til{Q}(\x_n)/\til{P}(\x_n)\to0$ folgt.
Wähle dazu $u\in C^\infi_0(\Om)$, $u\neq0$ fest und setze
\begin{equation}
u_n(x):=u(x)\frac{\e^{\i\ska{x}{\x_n}}}{\til{P}(\x_n)},\quad\text{f.a.}~u\in\R^n,
\end{equation}
für jedes $n\in\N$.
Wendet man die Leibnizsche Formel auf $u_n(x)$ an,
so erhält man
\begin{equation}
P(D)u_n(x)=\e^{\i\ska{x}{\x_n}}\sum_\al\frac{P^{(\al)}(\x_n)}{\til{P}(\x_n)}\frac{D^\al u(x)}{\al!}.
\end{equation}
Beachtet man, dass die Ausdrücke $P^{(\al)}/\til{P}$ stets beschränkt sind
und wendet wieder einmal Korollar \ref{cor:2:2.6} an,
so erhält man, dass
\begin{equation}\label{eq:2:PDunb}
\norm{P(D)u_n}\leq C,
\end{equation}
für ein $C\in(0,\infi)$.
Wir zerlegen nun
\begin{equation}\label{eq:2:kz}
\norm{Q(D)u_n-Q(D)u_m}^2=\norm{Q(D)u_n}^2+\norm{Q(D)u_m}^2-\de_{nm},
\end{equation}
wobei
\begin{equation}
\de_{nm}=2\Re\left\{\sum_{\al,\be}\frac{Q^{(\al)}(\x_n)}{\til{P}(\x_n)}\frac{\cc{Q^{(\be)}(\x_n)}}{\til{P}(\x_n)}
\frac1{\al!\be!}\int D^\al u\cc{D^\be u}\e^{\i\ska{x}{\x_n-\x_m}}\d x\right\}.
\end{equation}
Um $\de_{nm}\to0$ zu erreichen wählen wir die Folge $\x_n$ (z.B.~durch Übergang zu einer Teilfolge) so,
dass $\x_n-\x_m\to\infi$, für $n,m\to\infi$, $n\neq m$.
Der Faktor $D^\al u\cc{D^\be u}$ im Integral ist integrierbar.
Da auch die Faktoren vor dem Integral nach Voraussetzung beschränkt sind,
folgt $\de_{nm}\to0$ mit dem Riemann-Lebesgue-Lemma.

Aufgrund von \eqref{eq:2:PDunb} können wir zu einer Teilfolge übergehen,
so dass $Q(D)u_{n^\pr}$ konvergiert.
Dann gilt wegen \eqref{eq:2:kz} auch
\begin{equation}
\norm{Q(D)u_{n^\pr}}^2=\sum_{\al,\be}\frac{\abs{Q(\x_{n^\pr})}^2}{\til{P}(\x_{n^\pr})^2}\Psi_{\al\be}\to0.
\end{equation}
Daraus folgt auch $\til{Q}(\x_{n^\pr})^2/\til{P}(\x_{n^\pr})^2\to0$.
\end{proof}
 % Minimale Operatoren :: Tilmann Kleiner, Andreas Bitter
% !TEX root = main.tex
\chapter{Maximale Operatoren}
 % Maximale Operatoren :: Thomas Hamm
% !TEX root = main.tex
\chapter{Operatoren von reellem Haupttyp}
\cite{Hormander:1960a}
 % Operatoren von reellem Haupttyp :: Julian Mauersberger
% !TEX root = main.tex
\chapter{Ein unlösbarer Operator}
\cite{Lewy:1957}
\cite{Hormander:1960b} % Ein unloesbarer Operator :: Robin Lang
%
%
\part{Untere Schranken an Pseudodifferentialoperatoren}
%
% !TEX root = main.tex
\chapter{Einleitung}
Hier sollen die Grundlagen dafür gelegt werden, auch Operatoren mit variablen Koeffizienten richtig behandeln zu können. Die Darstellung basiert auf \cite{Hormander:1965}, \cite{Hormander:1966}, sowie \cite[Kapitel 18]{Hormander:1985}. Zuerst verallgemeinern wir den Begriff des Differentialoperators soweit, daß wir auch Inverse und allgemeinere Funktionen von solchen Operatoren behandeln können. 

\section{Operatoren und Symbole}
Sei $\Omega$ eine Mannigfaltigkeit. Differentialoperatoren auf $\Omega$ können dann in lokalen Karten definiert werden oder global durch ihre Eigenschaften charakterisiert werden. Die bekannteste davon ist, daß eine stetige lineare Abbildung $P:\rmC_0^\infty(\Omega)\to\rmC^\infty(\Omega)$ genau dann Differentialoperator ist, wenn $P$ lokal ist, also wenn
\begin{equation}
  \forall f\in\rmC_0^\infty(\Omega)\quad:\quad \supp Pf \subseteq \supp f
\end{equation}
gilt. Für Verallgemeinerungen brauchbarer ist folgende Charakterisierung:

\begin{lem}
Eine  lineare Abbildung   $P:\rmC_0^\infty(\Omega)\to\rmC^\infty(\Omega)$ ist genau dann ein Differentialoperator der Ordnung $m$, wenn für alle $f\in\rmC_0^\infty(\Omega)$ und alle $g\in\rmC^\infty(\Omega)$ die Funktion 
\begin{equation}\label{eq:6:6.2}
\e^{-\i\lambda g} P(f\e^{\i\lambda g}) = \sum_{j=0}^m P_j(f,g) \lambda^j
\end{equation}
ein Polynom vom Grad $m$ in $\lambda$ ist.
\end{lem}
\begin{proof}
Die Hinrichtung ist klar. Zum Beweis der Rückrichtung sei $x'\in\Omega$ ein Punkt und $f\in\rmC_0^\infty(\Omega)$ mit $f=1$  in einer Umgebung $\omega$ von $x'$. Sei weiter $\xi\in\R^n$ und $g(x) = x\cdot\xi$. Dann folgt aus \eqref{eq:6:6.2}
\begin{equation}
    \e^{-\i \lambda x\cdot\xi} P ( f\e^{\i \lambda x\cdot\xi}) = \sum_{j=0}^m p_j(f;x,\xi) \lambda^j
\end{equation}
mit $\rmC^\infty$-Funktionen auf $\omega\times\R^n$ als Koeffizienten $p_j(f;x,\xi)$. Da auch
$p_j(f;x,\lambda\xi)=p_j(f;x,\xi)\lambda^j$ gilt, ist $p_j(f;x,\xi)$ homogen vom Grad $j$. Die einzigen auf ganz $\R^n$ glatten homogenen Funktionen sind Polynome,
$p_j(f;x,\xi) = \sum_{|\alpha|=j} a_\alpha(f;x) \xi^\alpha$. Sei nun $u\in\rmC_0^\infty(\omega)$. Dann gilt mit der Fourierschen Inversionsformel
\begin{equation}
    u(x) = (2\pi)^{-n/2} \int \e^{\i x\cdot\xi} \widehat u(\xi)\d\xi
\end{equation}
und da $u=fu$ folgt insbesondere 
\begin{align}
    Pu(x) &= P(fu)(x) = (2\pi)^{-n/2} \int P(f \e^{\i x\cdot\xi}) \widehat u(\xi)\d\xi \notag \\&= \sum_{j=0}^m (2\pi)^{-n/2} \int \e^{\i x\cdot\xi} p_j(f;x,\xi) \widehat u(\xi)\d\xi
    = \sum_{|\alpha|\le m} a_\alpha(f;x)\D^\alpha u.
\end{align}
Mit Linearität folgt die Behauptung.
\end{proof}
 % Einleitung, Teil 2
% !TEX root = main.tex
\chapter{Die Ungleichung von G\r{a}rding}
\cite{Garding:1953}
\cite{Lax:1966}
  % Gardingsche Ungleichung :: Simon Barth
% !TEX root = main.tex
\chapter{Die Ungleichung von Melin}
\cite{Melin:1971} % Ungleichung von Melin :: Jonas Brinker

%
%

\chapter{Die Ungleichung von Feffermann und Phong}


% Literaturverzeichnis
\bibliographystyle{alpha}
\bibliography{seminar}
% Index
\printindex
\end{document}
